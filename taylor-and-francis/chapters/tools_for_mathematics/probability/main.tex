\chapter{Probability}
\label{\detokenize{tools-for-mathematics/06-probability/introduction/main:probability}}\label{\detokenize{tools-for-mathematics/06-probability/introduction/main:id1}}\label{\detokenize{tools-for-mathematics/06-probability/introduction/main::doc}}

Probability is the study of random events. Computers are particularly helpful
here as they can be used to carry out a number of experiments to confirm and/or
explore theoretic results.


In practice studying probability will often involve:
\begin{itemize}
\item 

calculating expected chances of an event occurring;

\item 

calculating the conditional chances of an event occurring given another event
occurring.

\end{itemize}


Here you will see how to instruct a computer to sample such events.



\begin{note}
In this chapter you will cover:
\begin{itemize}
\item 

Generating random numbers.

\item 

Randomly sample from a given collection of items.

\item 

Write python functions to be able to repeat experiments.

\end{itemize}
\end{note}




\section{Tutorial}
\label{\detokenize{tools-for-mathematics/06-probability/tutorial/main:tutorial}}\label{\detokenize{tools-for-mathematics/06-probability/tutorial/main::doc}}

You will solve the following problem using a computer to estimate the expected
probabilities:

An experiment consists of selecting a token from a bag and spinning a coin. The
bag contains 5 red tokens and 7 blue tokens. A token is selected at random from
the bag, its colour is noted and then the token is returned to the bag.


When a red token is selected, a biased coin with probability \(\frac{2}{3}\)
of landing heads is spun.


When a blue token is selected a fair coin is spun.
\begin{enumerate}

\item 

What is the probability of picking a red token?

\item 

What is the probability of obtaining Heads?

\item 

If a heads is obtained, what is the probability of having selected a red
token.

\end{enumerate}



You will use the \texttt{random} library from the Python standard library to do this.
First start off by building a Python \textbf{tuple} to represent the bag with the
tokens. Assign this to a variable \texttt{bag}:


\begin{pyin}
bag = (
    "Red",
    "Red",
    "Red",
    "Red",
    "Red",
    "Blue",
    "Blue",
    "Blue",
    "Blue",
    "Blue",
    "Blue",
    "Blue",
)
bag
\end{pyin}





\begin{raw}
('Red',
 'Red',
 'Red',
 'Red',
 'Red',
 'Blue',
 'Blue',
 'Blue',
 'Blue',
 'Blue',
 'Blue',
 'Blue')
\end{raw}






\begin{note}
You are using the circular brackets \texttt{()} and the quotation marks
\texttt{"}. Those are important and cannot be omitted. The choice of brackets
\texttt{()} as opposed to \texttt{\{\}} or \texttt{[]} is important as it instructs Python to do
different things. You can use \texttt{"} or \texttt{'}
interchangeably.
\end{note}



Instead of writing every copy of color you can create a Python \textbf{list} which allows
you to carry out some basic algebra on the items:
\begin{itemize}
\item 

Create a list with 5 \texttt{"Red"}s.

\item 

Create a list with 7 \texttt{"Blue"}s.

\item 

Combine both lists:

\end{itemize}




\begin{pyin}
bag = ["Red"] * 5 + ["Blue"] * 7
bag
\end{pyin}





\begin{raw}
['Red',
 'Red',
 'Red',
 'Red',
 'Red',
 'Blue',
 'Blue',
 'Blue',
 'Blue',
 'Blue',
 'Blue',
 'Blue']
\end{raw}

Now to sample from that use the \texttt{random} library which has a \texttt{choice}
command:

\begin{pyin}
import random

random.choice(bag)
\end{pyin}





\begin{raw}
'Blue'
\end{raw}

If you run this many times we will not always get the same outcome:

\begin{pyin}
random.choice(bag)
\end{pyin}





\begin{pyin}
'Blue'
\end{pyin}

The \texttt{bag} variable is unchanged:

\begin{pyin}
bag
\end{pyin}

\begin{pyin}
['Red',
 'Red',
 'Red',
 'Red',
 'Red',
 'Blue',
 'Blue',
 'Blue',
 'Blue',
 'Blue',
 'Blue',
 'Blue']
\end{pyin}





In order to answer the first question (what is the probability of picking a red
token) repeat this many times:
Do this by defining a Python function (which is akin to a mathematical
function) that makes repeating code possible:




\begin{pyin}
def pick_a_token(container):
    """
    A function to randomly sample from a `container`.
    """
    return random.choice(container)
\end{pyin}


You can then call this function, passing \texttt{bag} to it as the \texttt{container} from
which to pick:

\begin{pyin}
pick_a_token(container=bag)
\end{pyin}





\begin{raw}
'Blue'
\end{raw}









\begin{pyin}
pick_a_token(container=bag)
\end{pyin}





\begin{raw}
'Red'
\end{raw}






In order to simulate the probability of picking a red token repeat
this not once or twice but tens of thousands of times. You will do this using
something called a ``list comprehension'' which is akin to the mathematical
notation commonly used to create sets:

\begin{equation*}
\begin{split}
    S_1 = \{f(x)\text{ for }x\text{ in }S_2\}
\end{split}
\end{equation*}

\begin{pyin}
number_of_repetitions = 10000
samples = [pick_a_token(container=bag) for repetition in range(number_of_repetitions)]
samples
\end{pyin}





\begin{pyin}
['Red',
 'Red',
 'Red',
 ...
 'Blue',
 'Blue',
 'Red',
 'Blue',
 ]
\end{pyin}

You can confirm that we have the correct number of samples:

\begin{pyin}
len(samples)
\end{pyin}

\begin{raw}
10000
\end{raw}

\begin{note}
\texttt{len} is the Python tool to get the length of a given Python iterable.
\end{note}



Using this you can now use \texttt{==} (double \texttt{=}) to check how many of those 
samples are \text{Red}:

\begin{pyin}
sum(token == "Red" for token in samples) / number_of_repetitions
\end{pyin}

\begin{raw}
0.4071
\end{raw}

You have sampled a probability of around .41. The theoretic value is \(\frac{5}{5 +
7}\):


\begin{pyin}
5 / (5 + 7)
\end{pyin}





\begin{raw}
0.4166666666666667
\end{raw}






To answer the second question (What is the probability of obtaining Heads?) you
need to make use of another Python tool: an \texttt{if} statement. This will let
you
write a function that does precisely what is described in the problem:
\begin{itemize}
\item 

Choose a token;

\item 

Set the probability of flipping a given coin;

\item 

Select that coin.

\end{itemize}


\begin{note}
For the second random selection (flipping a coin) you will not choose from a list
but instead select a random number between 0 and 1.
\end{note}





\begin{pyin}
def sample_experiment(bag):
    """
    This samples a token from a given bag and then
    selects a coin with a given probability.

    If the sampled token is red then the probability
    of selecting heads is 2/3 otherwise it is 1/2.

    This function returns both the selected token
    and the coin face.
    """
    selected_token = pick_a_token(container=bag)

    if selected_token == "Red":
        probability_of_selecting_heads = 2 / 3
    else:
        probability_of_selecting_heads = 1 / 2

    if random.random() < probability_of_selecting_heads:
        coin = "Heads"
    else:
        coin = "Tails"
    return selected_token, coin
\end{pyin}





Using this you can sample according to the problem description:





\begin{pyin}
sample_experiment(bag=bag)
\end{pyin}





\begin{raw}
('Red', 'Heads')
\end{raw}


\begin{pyin}
sample_experiment(bag=bag)
\end{pyin}

\begin{raw}
('Red', 'Tails')
\end{raw}






You can now find out the probability of selecting heads by carrying out a large
number of repetitions and checking which ones have a coin that is heads:





\begin{pyin}
samples = [sample_experiment(bag=bag) for repetition in range(number_of_repetitions)]
sum(coin == "Heads" for token, coin in samples) / number_of_repetitions
\end{pyin}





\begin{raw}
0.576
\end{raw}






You can compute this theoretically as well, the expected probability is:




\begin{pyin}
import sympy as sym

sym.S(5) / (12) * sym.S(2) / 3 + sym.S(7) / (12) * sym.S(1) / 2
\end{pyin}




\begin{equation*}
\begin{split}\displaystyle \frac{41}{72}\end{split}
\end{equation*}







\begin{pyin}
41 / 72
\end{pyin}





\begin{raw}
0.5694444444444444
\end{raw}






You can also use the samples to calculate the conditional probability that a
token was read if the coin is heads. This is done again using the list
comprehension notation but including an \texttt{if} statement which 
emulates the mathematical notation:
\begin{equation*}
\begin{split}
    S_3 = \{x \in S_1  | \text{ if some property of \(x\) holds}\}
\end{split}
\end{equation*}




\begin{pyin}
samples_with_heads = [(token, coin) for token, coin in samples if coin == "Heads"]
sum(token == "Red" for token, coin in samples_with_heads) / len(samples_with_heads)
\end{pyin}





\begin{raw}
0.49236111111111114
\end{raw}






Using Bayes’ theorem this is given theoretically by:
\begin{equation*}
\begin{split}
    P(\text{Red}|\text{Heads}) = \frac{P(\text{Heads} | \text{Red})P(\text{Red})}{P(\text{Heads})}
\end{split}
\end{equation*}



\begin{pyin}
(sym.S(2) / 3 * sym.S(5) / 12) / (sym.S(41) / 72)
\end{pyin}




\begin{equation*}
\begin{split}\displaystyle \frac{20}{41}\end{split}
\end{equation*}







\begin{pyin}
20 / 41
\end{pyin}





\begin{raw}
0.4878048780487805
\end{raw}







\begin{note}
In this tutorial we have
\begin{itemize}
\item 

Randomly sampled from an iterable.

\item 

Randomly sampled a number between 0 and 1.

\item 

Written a function to represent a random experiment.

\item 

Created a list using list comprehensions.

\item 

Counted outcomes of random experiments.

\end{itemize}
\end{note}





\section{How to}
\label{\detokenize{tools-for-mathematics/06-probability/how/main:how-to}}\label{\detokenize{tools-for-mathematics/06-probability/how/main::doc}}

\subsection{Create a list}
\label{\detokenize{tools-for-mathematics/06-probability/how/main:create-a-list}}\label{\detokenize{tools-for-mathematics/06-probability/how/main:id1}}

To create a list which is an ordered collection of objects that \textbf{can} be
changed use the \texttt{{[}{]}} brackets.


\begin{pyin}
collection = [value_1, value_2, value_3, …, value_n]
\end{pyin}



For example:




\begin{pyin}
basket = ["Bread", "Biscuits", "Coffee"]
basket
\end{pyin}





\begin{raw}
['Bread', 'Biscuits', 'Coffee']
\end{raw}





You can insert an element to the end of a list by appending to it:




\begin{pyin}
basket.append("Tea")
basket
\end{pyin}





\begin{raw}
['Bread', 'Biscuits', 'Coffee', 'Tea']
\end{raw}





You can also combine lists together:




\begin{pyin}
other_basket = ["Toothpaste"]
basket = basket + other_basket
basket
\end{pyin}





\begin{pyin}
['Bread', 'Biscuits', 'Coffee', 'Tea', 'Toothpaste']
\end{pyin}





As for tuples we can also access elements using their indices:




\begin{pyin}
basket[3]
\end{pyin}





\begin{pyin}
'Tea'
\end{pyin}





\subsection{Define a function}
\label{\detokenize{tools-for-mathematics/06-probability/how/main:define-a-function}}\label{\detokenize{tools-for-mathematics/06-probability/how/main:id2}}

Define a function using the \texttt{def} keyword (short for define):


\begin{pyin}
def name(variable1, variable2, ...):
    """
    A docstring between triple quotation to describe what is happening
    """
    INDENTED BLOCK OF CODE
    return output
\end{pyin}



For example define \(f:\mathbb{R}\to\mathbb{R}\) given by \(f(x) = x ^
3\) using the following:




\begin{pyin}
def x_cubed(x):
    """
    A function to return x ^ 3
    """
    return x ** 3
\end{pyin}

It is important to include the \texttt{docstring} as this allows us to make sure our
code is clear. You can access that docstring using \texttt{help}:

\begin{pyin}
help(x_cubed)
\end{pyin}





\begin{raw}
Help on function x_cubed in module __main__:

x_cubed(x)
    A function to return x ^ 3
\end{raw}





\subsection{Call a function}
\label{\detokenize{tools-for-mathematics/06-probability/how/main:call-a-function}}

Once a function is defined call it using the \texttt{()}:


\begin{pyin}
name(variable1, variable2, ...)
\end{pyin}



For example:




\begin{pyin}
x_cubed(2)
\end{pyin}





\begin{raw}
8
\end{raw}







\begin{pyin}
x_cubed(5)
\end{pyin}





\begin{raw}
125
\end{raw}







\begin{pyin}
import sympy as sym

x = sym.Symbol("x")
x_cubed(x)
\end{pyin}




\begin{equation*}
\begin{split}\displaystyle x^{3}\end{split}
\end{equation*}




\subsection{Run code based on a condition}
\label{\detokenize{tools-for-mathematics/06-probability/how/main:conditional-running-of-code}}

To run code depending on whether or not a particular condition is met use
an \texttt{if} statement.


\begin{pyin}
if condition:
    INDENTED BLOCK OF CODE TO RUN IF CONDITION IS TRUE
else:
    OTHER INDENTED BLOCK OF CODE TO RUN IF CONDITION IS NOT TRUE
\end{pyin}



These \texttt{if} statements are most useful when combined with functions. For example
you can define the following function:
\begin{equation*}
\begin{split}
    f(x) = \begin{cases}
            x ^ 3&\text{ if }x < 0\\
            x ^ 2&\text{ otherwise}
            \end{cases}
\end{split}
\end{equation*}



\begin{pyin}
def f(x):
    """
    A function that returns x ^ 3 if x is negative.
    Otherwise it returns x ^ 2.
    """
    if x < 0:
        return x ** 3
    return x ** 2
\end{pyin}







\begin{pyin}
f(0)
\end{pyin}





\begin{raw}
0
\end{raw}







\begin{pyin}
f(-1)
\end{pyin}





\begin{raw}
-1
\end{raw}







\begin{pyin}
f(3)
\end{pyin}





\begin{raw}
9
\end{raw}





Here is another example of a function that returns the price of a given item, if
the item is not specific in the function then the price is 0:




\begin{pyin}
def get_price_of_item(item):
    """
    Returns the price of an item:

    - 'Bread': 2
    - 'Biscuits': 3
    - 'Coffee': 1.80
    - 'Tea': .50
    - 'Toothpaste': 3.50

    Other items will give a price of 0.
    """
    if item == "Bread":
        return 2
    if item == "Biscuits":
        return 3
    if item == "Coffee":
        return 1.80
    if item == "Tea":
        return 0.50
    if item == "Toothpaste":
        return 3.50
    return 0
\end{pyin}







\begin{pyin}
get_price_of_item("Toothpaste")
\end{pyin}





\begin{raw}
3.5
\end{raw}







\begin{pyin}
get_price_of_item("Biscuits")
\end{pyin}





\begin{raw}
3
\end{raw}







\begin{pyin}
get_price_of_item("Rollerblades")
\end{pyin}





\begin{raw}
0
\end{raw}





\subsection{Create a list using a list comprehension}
\label{\detokenize{tools-for-mathematics/06-probability/how/main:create-a-list-using-a-list-comprehension}}\label{\detokenize{tools-for-mathematics/06-probability/how/main:id3}}

You can create a new list from an old list using a \textbf{list comprehension}.


\begin{pyin}
collection = [f(item) for item in iterable]
\end{pyin}



This corresponds to building a set from another set in the usual mathematical
notation:
\begin{equation*}
\begin{split}
S_2 = \{f(x)\text{ for x in }S_1\}
\end{split}
\end{equation*}

If \(f(x)=x - 5\) and \(S_1=\{2, 5, 10\}\) then we would have:
\begin{equation*}
\begin{split}
S_2 = \{-3, 0, 5\}
\end{split}
\end{equation*}

In Python this is done as follows:

\begin{pyin}
new_list = [object for object in old_list]
\end{pyin}




\begin{pyin}
s_1 = [2, 5, 10]
s_2 = [x - 5 for x in s_1]
s_2
\end{pyin}





\begin{raw}
[-3, 0, 5]
\end{raw}





You can combine this with functions to write succinct efficient code.


For example you can compute the price of a basket of goods using the following:




\begin{pyin}
basket = ["Tea", "Tea", "Toothpaste", "Bread"]
prices = [get_price_of_item(item) for item in basket]
prices
\end{pyin}





\begin{raw}
[0.5, 0.5, 3.5, 2]
\end{raw}





\subsection{Add items in a list}
\label{\detokenize{tools-for-mathematics/06-probability/how/main:adding-items-in-a-list}}

You can compute the sum of items in a list using the \texttt{sum} tool:




\begin{pyin}
sum([1, 2, 3])
\end{pyin}





\begin{raw}
6
\end{raw}





You can also directly use the \texttt{sum} without specifically creating the list. This
corresponds to the following mathematical notation:

\begin{equation*}
\begin{split}
    \sum_{s\in S}f(s)
\end{split}
\end{equation*}

and is done using the following:

\begin{pyin}
sum(f(object) for object in old_list)
\end{pyin}


This gives the same result as:

\begin{pyin}
sum([f(object) for object in old_list])
\end{pyin}


but it is more efficient.
Here is an example of getting the total price of a basket of goods:




\begin{pyin}
basket = ["Tea", "Tea", "Toothpaste", "Bread"]
total_price = sum(get_price_of_item(item) for item in basket)
total_price
\end{pyin}





\begin{raw}
6.5
\end{raw}





\subsection{Sample from an iterable}
\label{\detokenize{tools-for-mathematics/06-probability/how/main:sample-from-an-iterable}}

To randomly sample from any collection of items
use the random library and the \texttt{choice} tool.


\begin{pyin}
random.choice(collection)
\end{pyin}






\begin{pyin}
import random

basket = ["Tea", "Tea", "Toothpaste", "Bread"]
random.choice(basket)
\end{pyin}





\begin{raw}
'Toothpaste'
\end{raw}






\subsection{Sample a random number}
\label{\detokenize{tools-for-mathematics/06-probability/how/main:sample-a-random-number}}

To sample a random number between 0 and 1 use the random library and the
\texttt{random} tool.


\begin{pyin}
random.random()
\end{pyin}



For example:





\begin{pyin}
import random

random.random()
\end{pyin}





\begin{raw}
0.7558634290782174
\end{raw}






\subsection{Reproduce random events}
\label{\detokenize{tools-for-mathematics/06-probability/how/main:reproduce-random-events}}\label{\detokenize{tools-for-mathematics/06-probability/how/main:id4}}

The random numbers processes generated by the Python random module are what are
called pseudo random which means that it is possible to get a computer to reproduce them by
\textbf{seeding} the random process.


\begin{pyin}
random.seed(int)
\end{pyin}





\begin{pyin}
import random

random.seed(0)
random.random()
\end{pyin}





\begin{raw}
0.8444218515250481
\end{raw}







\begin{pyin}
random.random()
\end{pyin}





\begin{raw}
0.7579544029403025
\end{raw}







\begin{pyin}
random.seed(0)
random.random()
\end{pyin}





\begin{raw}
0.8444218515250481
\end{raw}


\section{Exercises}
\label{\detokenize{tools-for-mathematics/06-probability/exercises/main:exercises}}\label{\detokenize{tools-for-mathematics/06-probability/exercises/main::doc}}

\begin{enumerate}

\item 

For each of the following, write a function, and repeatedly use it to simulate
the probability of an event occurring with the following chances:
\begin{enumerate}

\item 

\(\frac{2}{7}\)

\item 

\(\frac{1}{10}\)

\item 

\(\frac{1}{100}\)

\item 

\(1\)

\end{enumerate}

\item 

Write a function, and repeatedly use it to simulate the probability of
selecting a red token from each of the following configurations:
\begin{enumerate}

\item 

A bag with 4 red tokens and 4 green tokens.

\item 

A bag with 4 red tokens, 4 green tokens and 10 yellow tokens.

\item 

A bag with 0 red tokens, 4 green tokens and 10 yellow tokens.

\end{enumerate}

\item 

An experiment consists of selecting a token from a bag and spinning a coin. The bag contains 3 red tokens and 4 blue tokens. A token is selected at random from the bag, its colour is noted and then the token is returned to the bag.


When a red token is selected, a biased coin with probability \(\frac{4}{5}\) of landing heads is spun.


When a blue token is selected, a biased coin with probability \(\frac{2}{5}\) of landing heads is spun.
\begin{enumerate}

\item 

Approximate the probability of picking a red token?

\item 

Approximate the probability of obtaining Heads?

\item 

If a heads is obtained, approximate the probability of having selected a red token.

\end{enumerate}

\item 

On a randomly chose day, the probability of an individual travelling to school by car, bicycle or on foot is \(1/2\), \(1/6\) and \(1/3\) respectively. The probability of being late when using these methods of travel is \(1/5\), \(2/5\) and \(1/10\) respectively.
\begin{enumerate}

\item 

Approximate the probability that an individual travels by foot and is late.

\item 

Approximate the probability that an individual is not late.

\item 

Given that an individual is late, approximate the probability that they did not travel on foot.

\end{enumerate}

\end{enumerate}


\section{Further information}
\label{\detokenize{tools-for-mathematics/06-probability/why/main:further-information}}\label{\detokenize{tools-for-mathematics/06-probability/why/main::doc}}

\subsection{What is the difference between a Python list and a Python tuple?}
\label{\detokenize{tools-for-mathematics/06-probability/why/main:what-is-the-difference-between-a-python-list-and-a-python-tuple}}\label{\detokenize{tools-for-mathematics/06-probability/why/main:difference-between-a-list-and-a-tuple}}

Two of the most used Python iterables are lists and tuples. In practice they
have a number of similarities, they are both ordered collections of objects that
can be used in list comprehensions as well as in other ways.
\begin{itemize}
\item 

Tuples are \textbf{immutable}

\item 

Lists are \textbf{mutable}

\end{itemize}


This means that once created tuples cannot be changed and lists can.


As a general rule of thumb: if you do not need to modify your iterable then use
a tuple as they are more computationally efficient.



\subsection{Why does the sum of booleans count the \texttt{True}s?}
\label{\detokenize{tools-for-mathematics/06-probability/why/main:why-does-the-sum-of-booleans-counts-the-trues}}

In the tutorial and elsewhere you created a list of booleans and then took the
sum. Here are some of the steps:




\begin{pyin}
samples = ("Red", "Red", "Blue")
\end{pyin}







\begin{pyin}
booleans = [sample == "Red" for sample in samples]
booleans
\end{pyin}





\begin{raw}
[True, True, False]
\end{raw}





When you take the \texttt{sum} of that list you get a numeric value:




\begin{pyin}
sum(booleans)
\end{pyin}





\begin{raw}
2
\end{raw}





This has in fact counted the \texttt{True} values as 1 and the \texttt{False} values as 0.




\begin{pyin}
int(True)
\end{pyin}





\begin{raw}
1
\end{raw}







\begin{pyin}
int(False)
\end{pyin}





\begin{raw}
0
\end{raw}





\subsection{What is the difference between \texttt{print} and \texttt{return}?}
\label{\detokenize{tools-for-mathematics/06-probability/why/main:what-is-the-difference-between-print-and-return}}

In functions you use the \texttt{return} statement. This does two things:
\begin{enumerate}

\item 

Assigns a value to the function run;

\item 

Ends the function.

\end{enumerate}


The \texttt{print} statement \textbf{only} displays the output.


As an example create the following set:
\begin{equation*}
\begin{split}
    S = \{f(x)\text{ for }x \in \{0, \pi / 4, \pi / 2, 3\pi / 4\}\}
\end{split}
\end{equation*}

where \(f(x)= \cos^2(x)\).


The correct way to do this is:




\begin{pyin}
import sympy as sym


def f(x):
    """
    Return the square of the cosine of x
    """
    return sym.cos(x) ** 2


S = [f(x) for x in (0, sym.pi / 4, sym.pi / 2, 3 * sym.pi / 4)]
S
\end{pyin}





\begin{raw}
[1, 1/2, 0, 1/2]
\end{raw}





If you replaced the \texttt{return} statement in the function definition with a
\texttt{print} you obtain:




\begin{pyin}
def f(x):
    """
    Return the square of the cosine of x
    """
    print(sym.cos(x) ** 2)


S = [f(x) for x in (0, sym.pi / 4, sym.pi / 2, 3 * sym.pi / 4)]
\end{pyin}





\begin{raw}
1
1/2
0
1/2
\end{raw}





The function has been run and it displays the output.


\textbf{However} if you look at what \texttt{S} is you see that the function has not returned
anything:




\begin{pyin}
S
\end{pyin}





\begin{raw}
[None, None, None, None]
\end{raw}







\subsection{How does Python sample randomness?}
\label{\detokenize{tools-for-mathematics/06-probability/why/main:how-does-python-sample-randomness}}

When using the Python random module you are in fact generating a pseudo random
process. True randomness is actually not common.


Pseudo randomness is an important area of mathematics as strong algorithms that
create unpredictable sequences of numbers are vital to cryptographic security.


The specific algorithm using in Python for randomness is called the Mersenne
twister algorithm is state of the art.


\subsection{What is the difference between a docstring and a comment}
\label{\detokenize{tools-for-mathematics/06-probability/why/main:what-is-the-difference-between-a-docstring-and-a-comment}}\label{\detokenize{tools-for-mathematics/06-probability/why/main:difference-between-a-docstring-and-a-comment}}

In Python it is possible to write statements that are ignored using the \texttt{\#}
symbol. This creates something called a “comment”. For example:




\begin{pyin}
# create a list to represent the tokens in a bag
bag = ["Red", "Red", "Blue"]
\end{pyin}

A docstring however is something that is “attached” to a function and can be
accessed by Python.
If you rewrite the function to sample the experiment of the tutorial without a
docstring but using comments we will have:





\begin{pyin}
def sample_experiment(bag):
    # Select a token
    selected_token = pick_a_token(container=bag)

    # If the token is red then the probability of selecting heads is 2/3
    if selected_token == "Red":
        probability_of_selecting_heads = 2 / 3
    # Otherwise it is 1 / 2
    else:
        probability_of_selecting_heads = 1 / 2

    # Select a coin according to the probability.
    if random.random() < probability_of_selecting_heads:
        coin = "Heads"
    else:
        coin = "Tails"

    # Return both the selected token and the coin.
    return selected_token, coin
\end{pyin}






Now if you try to access the help for the function you will not get it:




\begin{pyin}
help(sample_experiment)
\end{pyin}





\begin{raw}
Help on function sample_experiment in module __main__:

sample_experiment(bag)
\end{raw}


Furthermore, if you look at the code with comments you will see that because of
the choice of variable names the comments are in fact redundant.

In software engineering it is generally accepted that comments indicate that
your code is not clear and so it is preferable to write clear documentation
explaining why something is done through docstrings.




\begin{pyin}
def sample_experiment(bag):
    """
    This samples a token from a given bag and then
    selects a coin with a given probability.

    If the sampled token is red then the probability
    of selecting heads is 2/3 otherwise it is 1/2.

    This function returns both the selected token
    and the coin face.
    """
    selected_token = pick_a_token(container=bag)

    if selected_token == "Red":
        probability_of_selecting_heads = 2 / 3
    else:
        probability_of_selecting_heads = 1 / 2

    if random.random() < probability_of_selecting_heads:
        coin = "Heads"
    else:
        coin = "Tails"
    return selected_token, coin
\end{pyin}

\chapter*{Preface}


Welcome to this book.

This is not a book for learning to program to do mathematics. There are
many excellent books that do this~\cite{bautista2014mathematics, morley2022applying,  saha2015doing}. This is a book for people who would
like to learn how to use programming tools to help do Mathematics.

Mathematics is often thought of as solving problems. In high school this can
be sets of quadratic equations that need to be solved or probabilities of
specific hands of cards that need to be calculated.

As one progresses further in to mathematics, the subject becomes less about
solving problems through mechanical calculation and more about using our
mathematical knowledge and insight to choose \textbf{which problems to solve}.

This is what this book attempts to address. It aims to be a user guide for how
the Python programming language can be used to reduce mechanical calculation
which leaves more space to do real mathematics.

Whilst no book should ever try to stop a mathematician from picking up a pen and
pencil and thinking about a problem, this one does aim to show how modern
mathematicians can replace some of the use of their pen with openly available
Python tools. For example, in Chapter~\ref{chp:algebra} how to solve an equation by essentially
just writing it down is covered. In Chapter~\ref{chp:probability} probabilities
of specific events are simulated.

In the second part of this book a more traditional approach of programming with
Python is used to show how to build tools. Not only does this cover commonly
taught programming techniques but also goes in to principles of software
development used in industry. For example, Chapter~\ref{chp:testing} covers how
to write code that tests software and Chapter~\ref{chp:documentation} covers a
modern way of writing documentation for software.

This book is for you, whether you are a seasoned professional mathematician who
would like to know some of the best practices for using Python or perhaps more
typically, if you are a first year University student with an understanding of
the mathematical topics covered


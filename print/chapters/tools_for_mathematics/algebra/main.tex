\chapter{Algebra}
\label{chp:algebra}
%\label{\detokenize{tools-for-mathematics/02-algebra/introduction/main:algebra}}\label{\detokenize{tools-for-mathematics/02-algebra/introduction/main::doc}}

\section{Introduction}
%\label{\detokenize{tools-for-mathematics/02-algebra/introduction/main:introduction}}

A typical secondary school curriculum includes Algebra which
is described, in the A-level syllabus
as:

\begin{quote}
``Algebra: this is an essential tool which supports and expresses mathematical
reasoning and provides a means to generalise across a number of contexts.''
\end{quote}

In practice this often means:
\begin{itemize}
\item Manipulating numeric expressions;
\item Manipulating symbolic expressions;
\item Solving equations.
\end{itemize}


You can use a computer to carry out some of these techniques.

\begin{note}
This chapter covers:

\begin{itemize}
\item Manipulating numeric and symbolic expressions.
\item Solving equations.
\end{itemize}
\end{note}

\section{Tutorial}
%\label{\detokenize{tools-for-mathematics/02-algebra/tutorial/main:tutorial}}\label{\detokenize{tools-for-mathematics/02-algebra/tutorial/main:algebra-tutorial}}\label{\detokenize{tools-for-mathematics/02-algebra/tutorial/main::doc}}

To demonstrate the ways in which a computer can assist with Algebra this
tutorial
will solve the following two problems:

\begin{enumerate}

\item 
Rationalise the denominator of \(\frac{1}{\sqrt{2} + 1}\)

\item 
Consider the quadratic: \(f(x)=2x ^ 2 + x + 1\):

\begin{enumerate}

\item Calculate the discriminant of the quadratic equation \(2x ^ 2 + x + 1 =
0\). What does this tell us about the solutions to the equation? What
does this tell us about the graph of \(f(x)\)?

\item By completing the square, show that the minimum point of \(f(x)\) is
\(\left(-\frac{1}{4}, \frac{7}{8}\right)\)

\end{enumerate}

\end{enumerate}


To do this, a specific collection of tools available in Python will be used.
Often specific sets of tools are separated in to things called \textbf{libraries}. 
Start by telling Python that you want to use the specific library for \textbf{symbolic
mathematics}:

\begin{pyin}
import sympy
\end{pyin}

This allows you to solve the first part of the question. First, create a
variable \texttt{expression} and \textbf{assign} it a value of \(\frac{1}{\sqrt{2} + 1}\).

\begin{pyin}
expression = 1 / (sympy.sqrt(2) + 1)
expression
\end{pyin}

\[\frac{1}{1 + \sqrt{2}}\]

\begin{note}
This is not what would happen if you plugged the above in to a basic calculator,
it would instead give us a value of:
\end{note}


\begin{pyin}
float(expression)
\end{pyin}

\[0.41421356237309503\]

The \texttt{sympy} library has a diverse set of tools available, one of which is to
algorithmically attempt to simplify an expression. Here is how to do that:

\begin{pyin}
sympy.simplify(expression)
\end{pyin}

\[ -1 + \sqrt{2}\]

This implies that:

\begin{equation*}
\begin{split}
    \frac{1}{\sqrt{2} + 1} = -1 + \sqrt{2}
\end{split}
\end{equation*}

Multiplying both sides by \({\sqrt{2} + 1}\) gives:

\[
1=\frac{1}{\sqrt{2} + 1}\times \left(\sqrt{2} + 1\right) = \left(-1 + \sqrt{2}\right)\times \left(\sqrt{2} + 1\right)
\]

The \texttt{sympy.simplify} command did not give much insight in to what happened
but you can confirm the above manipulation by expanding \(\left(-1 +
\sqrt{2}\right)\times \left(\sqrt{2} + 1\right)\).
Here is how to do that:

\begin{pyin}
sympy.expand((-1 + sympy.sqrt(2)) * (1 + sympy.sqrt(2)))
\end{pyin}

\[
1
\]

The \texttt{sympy} library allows you to carry out basic expression manipulation.
Now consider the second part of the question:

\begin{enumerate}

\item 

Consider the quadratic: \(f(x)=2x ^ 2 + x + 1\):

\item 

Calculate the discriminant of the quadratic equation \(2x ^ 2 + x + 1 =
0\). What does this tell us about the solutions to the equation? What
does this tell us about the graph of \(f(x)\)?

\item 

By completing the square, show that the minimum point of \(f(x)\) is
\(\left(-\frac{1}{4}, \frac{7}{8}\right)\)

\end{enumerate}


Start by reassigning the value of the variable \texttt{expression} to be the
expression: \(2x ^ 2 + x + 1\).

\begin{pyin}
x = sympy.Symbol("x")
expression = 2 * x ** 2 + x + 1
expression
\end{pyin}

\[
2 x^{2} + x + 1
\]


The first line communicates to the code that \texttt{x} is going to be a symbolic variable.

\begin{note}
\textbf{Recall} that the \texttt{**} symbol is how you communicate exponentiation.
\end{note}

You can immediately use this to compute the discriminant:

\begin{pyin}
sympy.discriminant(expression)
\end{pyin}

\[-7\]

Now, complement this with mathematical knowledge: if a quadratic has a
negative discriminant then it does not have any roots and all the values are of
the same sign as the coefficient of \(x ^ 2 \). Which in this case is \(2>0\).
Confirm this by directly creating the equation. Do this by creating a
variable \texttt{equation} and assigning it the equation which has a
\texttt{lhs} and a \texttt{rhs}:

\begin{pyin}
equation = sympy.Eq(lhs=expression, rhs=0)
equation
\end{pyin}

\[2 x^{2} + x + 1 = 0\]

Now ask \texttt{sympy} to solve it:

\begin{pyin}
sympy.solveset(equation)
\end{pyin}

\[\left\{- \frac{1}{4} - \frac{\sqrt{7} i}{4}, - \frac{1}{4} + \frac{\sqrt{7} i}{4}\right\}\]

Indeed the only solutions are imaginary numbers: this confirms that the graph of
\(f(x)\) is a convex parabola that is above the \(y=0\) line.
Now complete the square so that you can write:

\begin{equation*}
\begin{split}
    f(x) = a (x - b) ^ 2 + c
\end{split}
\end{equation*}

for some values of \(a, b, c\).
Create variables that have those 3 constants as value but also create a variable \texttt{completed\_square} and assign it the general expression:

\begin{pyin}
a, b, c = sympy.Symbol("a"), sympy.Symbol("b"), sympy.Symbol("c")
completed_square = a * (x - b) ** 2 + c
completed_square
\end{pyin}

\[a \left(- b + x\right)^{2} + c\]

Expand this:

\begin{pyin}
sympy.expand(completed_square)
\end{pyin}

\[a b^{2} - 2 a b x + a x^{2} + c\]

Use \texttt{sympy} to solve the various equations that arise from comparing
the coefficients of:

\begin{equation*}
\begin{split}
    f(x) = 2x ^2 + x + 1
\end{split}
\end{equation*}

with the completed square.
First, you see that the coefficient of \(x ^ 2\) gives us an equation:

\begin{equation*}
\begin{split}
    a = 2
\end{split}
\end{equation*}

For completeness write the code that solves this trivial equation:

\begin{pyin}
equation = sympy.Eq(a, 2)
sympy.solveset(equation, a)
\end{pyin}

\begin{equation*}
\begin{split}\displaystyle \left\{2\right\}\end{split}
\end{equation*}

Now substitute this value of \(a\) in to the completed square and update the variable with the new value:

\begin{pyin}
completed_square = completed_square.subs({a: 2})
completed_square
\end{pyin}

\begin{equation*}
\begin{split}\displaystyle c + 2 \left(- b + x\right)^{2}\end{split}
\end{equation*}

\begin{note}
The different types of brackets being used there: both \texttt{()} and \texttt{\{\}}. This is
important and has specific meaning in Python which will be covered in future
chapters.
\end{note}

Now look at the expression with the two remaining constants:

\begin{pyin}
sympy.expand(completed_square)
\end{pyin}

\begin{equation*}
\begin{split}\displaystyle 2 b^{2} - 4 b x + c + 2 x^{2}\end{split}
\end{equation*}

Comparing the coefficients of \(x\) gives:

\begin{equation*}
\begin{split}
  - 4 b = 1
\end{split}
\end{equation*}

\begin{pyin}
equation = sympy.Eq(-4 * b, 1)
sympy.solveset(equation, b)
\end{pyin}

\begin{equation*}
\begin{split}\displaystyle \left\{- \frac{1}{4}\right\}\end{split}
\end{equation*}

Substitute this value of \(b\) back in to our expression.
Make a point to tell sympy to treat \(1 / 4\) symbolically and to not
calculate the numeric value:

\begin{pyin}
completed_square = completed_square.subs({b: -1 / sympy.S(4)})
completed_square
\end{pyin}

\begin{equation*}
\begin{split}\displaystyle c + 2 \left(x + \frac{1}{4}\right)^{2}\end{split}
\end{equation*}

Expand this to see the expression with the one remaining constant gives:

\begin{pyin}
sympy.expand(completed_square)
\end{pyin}

\begin{equation*}
\begin{split}\displaystyle c + 2 x^{2} + x + \frac{1}{8}\end{split}
\end{equation*}

This gives a final equation for the constant term:

\begin{equation*}
\begin{split}
    c + 1 / 8 = 1
\end{split}
\end{equation*}

Now use sympy to find the value of \(c\):

\begin{pyin}
sympy.solveset(sympy.Eq(c + sympy.S(1) / 8, 1), c)
\end{pyin}

\begin{equation*}
\begin{split}\displaystyle \left\{\frac{7}{8}\right\}\end{split}
\end{equation*}

As before substitute in and update the value of \texttt{completed\_square}:

\begin{pyin}
completed_square = completed_square.subs({c: 7 / sympy.S(8)})
completed_square
\end{pyin}

\begin{equation*}
\begin{split}\displaystyle 2 \left(x + \frac{1}{4}\right)^{2} + \frac{7}{8}\end{split}
\end{equation*}

Using this shows that the there are indeed no values of \(x\) which give
negative values of \(f(x)\) as \(f(x)\) is a square added to a constant.
The minimum is when \(x=-1/4\) which gives: \(f(-1/4)=7/8\):

\begin{pyin}
completed_square.subs({x: -1 / sympy.S(4)})
\end{pyin}

\begin{equation*}
\begin{split}\displaystyle \frac{7}{8}\end{split}
\end{equation*}

\begin{note}
This tutorial has:
\begin{itemize}
\item 

Created symbolic expressions.

\item 

Obtained approximate values for numerical symbolic expressions.

\item 

Expanded and simplified symbolic expressions.

\item 

Created symbolic equations.

\item 

Solve symbolic equations.

\item 

Substituted values in to symbolic expressions.

\end{itemize}
\end{note}

\section{How to}

\subsection{Create a symbolic numeric value}

To create a symbolic numerical value use \texttt{sympy.S}.

\begin{api}
sympy.S(a)
\end{api}


For example:




\begin{pyin}
import sympy

value = sympy.S(3)
value
\end{pyin}




\begin{equation*}
\begin{split}\displaystyle 3\end{split}
\end{equation*}

If you combine a symbolic value with a non symbolic value it will automatically
give a symbolic value:

\begin{pyin}
1 / value
\end{pyin}

\begin{equation*}
\begin{split}\displaystyle \frac{1}{3}\end{split}
\end{equation*}

\subsection{Get the numerical value of a symbolic expression}
\label{\detokenize{tools-for-mathematics/02-algebra/how/main:how-to-get-the-numerical-value-of-a-symbolic-expression}}

You can get the numerical value of a symbolic value using \texttt{float} or
\texttt{int}:
\begin{itemize}
\item 

\texttt{float} will give the numeric approximation in \(\mathbb\{R\}\)

\begin{api}
float(x)
\end{api}

\item 

\texttt{int} will give the integer value

\begin{api}
int(x)
\end{api}

\end{itemize}


For example, to create a symbolic numeric variable with value
\(frac{1}{5}\):

\begin{pyin}
value = 1 / sympy.S(5)
value
\end{pyin}




\begin{equation*}
\begin{split}\displaystyle \frac{1}{5}\end{split}
\end{equation*}




To get the numerical value:




\begin{pyin}
float(value)
\end{pyin}





\begin{raw}
0.2
\end{raw}





To get the integer part:




\begin{pyin}
int(value)
\end{pyin}

\begin{raw}
0
\end{raw}

\begin{note}
This is not rounding to the nearest integer. It is returning the integer part.
\end{note}


\subsection{Factor an expression}

Use the \texttt{sympy.factor} tool to factor expressions.

\begin{pyin}
sympy.factor(expression)
\end{pyin}

For example:

\begin{pyin}
x = sympy.Symbol("x")
sympy.factor(x ** 2 - 9)
\end{pyin}


\begin{equation*}
\begin{split}\displaystyle \left(x - 3\right) \left(x + 3\right)\end{split}
\end{equation*}

\subsection{Expand an expression}

Use the \texttt{sympy.expand} tool to expand expressions.

\begin{pyin}
sympy.expand(expression)
\end{pyin}


For example:




\begin{pyin}
sympy.expand((x - 3) * (x + 3))
\end{pyin}




\begin{equation*}
\begin{split}\displaystyle x^{2} - 9\end{split}
\end{equation*}




\subsection{Simplify an expression}
%\label{\detokenize{tools-for-mathematics/02-algebra/how/main:how-to-simplify-an-expression}}

Use the \texttt{sympy.simplify} tool to simplify an expression.

\begin{pyin}
sympy.simplify(expression)
\end{pyin}


For example:




\begin{pyin}
sympy.simplify((x - 3) * (x + 3))
\end{pyin}




\begin{equation*}
\begin{split}\displaystyle x^{2} - 9\end{split}
\end{equation*}

\begin{note}
This will not always give the expected (or any) result. At times it could be
more beneficial to use \texttt{sympy.expand} and/or \texttt{sympy.factor}.
\end{note}

\subsection{How to solve an equation}
%\label{\detokenize{tools-for-mathematics/02-algebra/how/main:how-to-solve-an-equation}}

Use the \texttt{sympy.solveset} tool to solve an equation. It takes two values as
inputs. The first is either:
\begin{itemize}
\item 

An expression for which a root is to be found

\item 

An equation

\end{itemize}


The second is the variable you want to solve for.

\begin{api}
sympy.solveset(equation, variable)
\end{api}


Here is how you can use \texttt{sympy} to obtain the roots of the general quadratic:

\begin{equation*}
\begin{split}
a x ^ 2 + bx + c
\end{split}
\end{equation*}



\begin{pyin}
a = sympy.Symbol("a")
b = sympy.Symbol("b")
c = sympy.Symbol("c")
quadratic = a * x ** 2 + b * x + c
sympy.solveset(quadratic, x)
\end{pyin}




\begin{equation*}
\begin{split}\displaystyle \left\{- \frac{b}{2 a} - \frac{\sqrt{- 4 a c + b^{2}}}{2 a}, - \frac{b}{2 a} + \frac{\sqrt{- 4 a c + b^{2}}}{2 a}\right\}\end{split}
\end{equation*}




Here is to solve the same equation but not for \(x\) but for
\(b\):




\begin{pyin}
sympy.solveset(quadratic, b)
\end{pyin}




\begin{equation*}
\begin{split}\displaystyle \left\{- \frac{a x^{2} + c}{x}\right\}\end{split}
\end{equation*}

It is however clearer to specifically write the equation to solve:

\begin{pyin}
equation = sympy.Eq(a * x ** 2 + b * x + c, 0)
sympy.solveset(equation, x)
\end{pyin}




\begin{equation*}
\begin{split}\displaystyle \left\{- \frac{b}{2 a} - \frac{\sqrt{- 4 a c + b^{2}}}{2 a}, - \frac{b}{2 a} + \frac{\sqrt{- 4 a c + b^{2}}}{2 a}\right\}\end{split}
\end{equation*}




\subsection{Substitute a value in to an expression}
\label{sec:substite_a_value_in_to_an_expression}
%\label{\detokenize{tools-for-mathematics/02-algebra/how/main:how-to-substitute-a-value-in-to-an-expression}}\label{\detokenize{tools-for-mathematics/02-algebra/how/main:id1}}

Given a \texttt{sympy} expression it is possible to substitute values in to it using
the \texttt{.subs()} tool.

\begin{api}
expression.subs({variable: value})
\end{api}

It is possible to pass multiple variables at a time.
For example to substitute the values for \(a, b, c\) in to the quadratic:

\begin{pyin}
quadratic = a * x ** 2 + b * x + c
quadratic.subs({a: 1, b: sympy.S(7) / 8, c: 0})
\end{pyin}

\begin{equation*}
\begin{split}\displaystyle x^{2} + \frac{7 x}{8}\end{split}
\end{equation*}

\section{Exercises}
%\label{\detokenize{tools-for-mathematics/02-algebra/exercises/main:exercises}}\label{\detokenize{tools-for-mathematics/02-algebra/exercises/main::doc}}

\textbf{After} completing the tutorial attempt the following exercises.


\textbf{If you are not sure how to do something, have a look at the “How To” section.}
\begin{enumerate}

\item 

Simplify the following expressions:
\begin{enumerate}

\item 

\(\frac{3}{\sqrt{3}}\)

\item 

\(\frac{2 ^ {78}}{2 ^ {12}2^{-32}}\)

\item 

\(8^0\)

\item 

\(a^4b^{-2}+a^{3}b^{2}+a^{4}b^0\)

\end{enumerate}

\item 

Solve the following equations:
\begin{enumerate}

\item 

\(x + 3 = -1\)

\item 

\(3 x ^ 2 - 2 x = 5\)

\item 

\(x (x - 1) (x + 3) = 0\)

\item 

\(4 x ^3 + 7x - 24 = 1\)

\end{enumerate}

\item 

Consider the equation: \(x ^ 2 + 4 - y = \frac{1}{y}\):
\begin{enumerate}

\item 

Find the solution to this equation for \(x\).

\item 

Obtain the specific solution when \(y = 5\). Do this in two ways:
substitute the value in to your equation and substitute the value in to
your solution.

\end{enumerate}

\item 

Consider the quadratic: \(f(x)=4x ^ 2 + 16x + 25\):
\begin{enumerate}

\item 

Calculate the discriminant of the quadratic equation \(4x ^ 2 + 16x + 25 =
0\). What does this tell us about the solutions to the equation? What
does this tell us about the graph of \(f(x)\)?

\item 

By completing the square, show that the minimum point of \(f(x)\) is
\(\left(-2, 9\right)\)

\end{enumerate}

\item 

Consider the quadratic: \(f(x)=-3x ^ 2 + 24x - 97\):
\begin{enumerate}

\item 

Calculate the discriminant of the quadratic equation \(-3x ^ 2 + 24x - 97 =
0\). What does this tell us about the solutions to the equation? What
does this tell us about the graph of \(f(x)\)?

\item 

By completing the square, show that the maximum point of \(f(x)\) is
\(\left(4, -49\right)\)

\end{enumerate}

\item 

Consider the function \(f(x) = x^ 2 + a x + b\).
\begin{enumerate}

\item 

Given that \(f(0) = 0\) and \(f(3) = 0\) obtain the values of \(a\) and \(b\).

\item 

By completing the square confirm that graph of \(f(x)\) has a line of symmetry at \(x=\frac{3}{2}\)

\end{enumerate}

\end{enumerate}

\section{Further information}
\label{\detokenize{tools-for-mathematics/02-algebra/why/main:further-information}}\label{\detokenize{tools-for-mathematics/02-algebra/why/main::doc}}

\subsection{Why is some code in separate libraries?}
\label{\detokenize{tools-for-mathematics/02-algebra/why/main:why-is-some-code-in-separate-libraries}}

When you run the \texttt{import sympy} command you are telling Python that you want to use
a specific set of tools. You will see other examples of this throughout this
book.
One of the advantages of having code in libraries is that it is more efficient
for Python to only use what is needed.
There are two types of Python libraries:
\begin{itemize}
\item 

Those that are part of the so called “standard library”: these are part of
Python itself.

\item 

Those that are completely separate: \texttt{sympy} is one such example of this.

\end{itemize}


This separation allows for the development of tools to be not dependent on each
other. The developers of \texttt{sympy} do not need to coordinate with the developers
of Python to make new releases of the software.


\subsection{Why do I need to use sympy?}
\label{\detokenize{tools-for-mathematics/02-algebra/why/main:why-do-we-need-to-use-sympy}}

\texttt{sympy} is the library for symbolic mathematics. There are other python libraries
for carrying out mathematics in Python.
For example, compute the value of the following expression:
\begin{equation*}
\begin{split}
    (\sqrt{2} + 2) ^ 2 - 2
\end{split}
\end{equation*}

You could compute this using the \texttt{math} library (for the square root tool):

\begin{pyin}
import math

(math.sqrt(2) + 2) ** 2 - 2
\end{pyin}





\[
9.65685424949238
\]





You could also make use of the fact that you do not need a square root tool at all:
\begin{equation*}
\begin{split}
    (\sqrt{2} + 2) ^ 2 - 2 = (2 ^ {1 / 2} + 2) ^ 2 - 2
\end{split}
\end{equation*}



\begin{pyin}
(2 ** (1 / 2) + 2) ** 2 - 2
\end{pyin}





\[
9.65685424949238
\]


You see that in both those instances, you have a numeric value for the expression
that seems to be precise up to 14 decimal places.


However, that is \textbf{not} the exact value of that expression. The exact value of
the expression needs to be computed symbolically:




\begin{pyin}
import sympy

expression = (sympy.sqrt(2) + 2) ** 2 - 2
sympy.expand(expression)
\end{pyin}




\begin{equation*}
\begin{split}\displaystyle 4 + 4 \sqrt{2}\end{split}
\end{equation*}




This is one example of why \texttt{sympy} is an effective tool for mathematicians.
The other one seen in this chapter is being able to compute expressions with no
numerical value at all:




\begin{pyin}
a = sympy.Symbol("a")
b = sympy.Symbol("b")
sympy.factor(a ** 2 - b ** 2)
\end{pyin}




\begin{equation*}
\begin{split}\displaystyle \left(a - b\right) \left(a + b\right)\end{split}
\end{equation*}




\subsection{Why do I sometimes see \texttt{from sympy import *}?}
\label{\detokenize{tools-for-mathematics/02-algebra/why/main:why-do-i-sometimes-see-from-sympy-import}}

There a number of resources available from which you can learn to use \texttt{sympy}. In
some instances you will not see \texttt{import sympy} but instead you will see 
\texttt{from sympy import *}.


\textbf{This it not a good way to do it.}


What this does is taking all the tools inside of sympy and putting it at the
same level of all the other tools available to you.
The problem with doing this is that it no longer makes your code clear.
An example of this are trigonometric functions. These exist in a number of
libraries:

\begin{pyin}
import math
\end{pyin}







\begin{pyin}
import sympy
\end{pyin}







\begin{pyin}
sympy.cos(0)
\end{pyin}




\begin{equation*}
\begin{split}\displaystyle 1\end{split}
\end{equation*}






\begin{pyin}
math.cos(0)
\end{pyin}





\begin{raw}
1.0
\end{raw}





One of these tools allows you to carry out exact computations:




\begin{pyin}
sympy.cos(sympy.pi / 4)
\end{pyin}




\begin{equation*}
\begin{split}\displaystyle \frac{\sqrt{2}}{2}\end{split}
\end{equation*}






\begin{pyin}
math.cos(math.pi / 4)
\end{pyin}


\begin{raw}
0.7071067811865476
\end{raw}





If you chose to import all the functionality using \texttt{from sympy import *}
then you
cannot tell immediately which function you are using (except from its output):




\begin{pyin}
from sympy import *
\end{pyin}







\begin{pyin}
from math import *
\end{pyin}







\begin{pyin}
cos(pi / 4)
\end{pyin}



\begin{raw}
0.7071067811865476
\end{raw}





In that case the second import has overwritten the first.

\begin{note}
\textbf{It is never recommended to use} \texttt{import *} this makes your code less clear
and you are more likely to make mistakes when your code is not clear.
\end{note}


\subsection{How to extract a solution from the output of \texttt{sympy.solveset}?}
%\label{\detokenize{tools-for-mathematics/02-algebra/why/main:how-to-extract-a-solution-from-the-output-of-sympy-solveset}}

In some cases you might want to directly access the items in a solution set. For
example if consider the equation \((x - 1)(x -
2)\).

\begin{pyin}
import sympy

x = sympy.Symbol("x")
expression = (x - 1) * (x - 2)
equation = sympy.Eq(expression, 0)
set_of_solutions = sympy.solveset(equation, x)
set_of_solutions
\end{pyin}




\begin{equation*}
\begin{split}\displaystyle \left\{1, 2\right\}\end{split}
\end{equation*}




The \texttt{set\_of\_solutions} has value the \textbf{set} of solutions of the
equation. If you
wanted to access them directly you can use the following:




\begin{pyin}
tuple_of_solutions = set_of_solutions.args
tuple_of_solutions
\end{pyin}





\[
(1, 2)
\]

This creates a \textbf{finite} ordered tuple of the solutions. You can use concepts
that are covered in Chapter~\ref{chp:combinatorics}
access them directly. Because there are two roots you can use the following to
create two new variables:

\begin{pyin}
x1, x2 = tuple_of_solutions
\end{pyin}

Substitute these value directly in to the expression:

\begin{pyin}
expression.subs({x: x1})
\end{pyin}




\begin{equation*}
\begin{split}\displaystyle 0\end{split}
\end{equation*}






\begin{pyin}
expression.subs({x: x2})
\end{pyin}




\begin{equation*}
\begin{split}\displaystyle 0\end{split}
\end{equation*}




Note that this is not always possible to get a finite ordered tuple of the
solutions, for example there are some equations
where the set of solutions is an infinite set:




\begin{pyin}
equation = sympy.Eq(sym.cos(x / 5), 0)
set_of_solutions = sympy.solveset(equation, x)
set_of_solutions
\end{pyin}




\begin{equation*}
\begin{split}\displaystyle \left\{10 n \pi + \frac{5 \pi}{2}\; \middle|\; n \in \mathbb{Z}\right\} \cup \left\{10 n \pi + \frac{15 \pi}{2}\; \middle|\; n \in \mathbb{Z}\right\}\end{split}
\end{equation*}




\subsection{Why do I sometimes see \texttt{import sympy as sym}?}
\label{\detokenize{tools-for-mathematics/02-algebra/why/main:why-do-i-sometimes-see-import-sympy-as-sym}}

In some resources you will see that instead of \texttt{import sympy} people use:
\texttt{import sympy as sym}. This is called \textbf{aliasing}. This is common and takes
advantage of the fact that Python can import a library and give it an
alias/nickname at the same time:

\begin{api}
import <library> as <nickname>
\end{api}


So with sympy you can use:




\begin{pyin}
import sympy as sym

sym.cos(sym.pi / 4)
\end{pyin}




\begin{equation*}
\begin{split}\displaystyle \frac{\sqrt{2}}{2}\end{split}
\end{equation*}




There is nothing stopping you using whatever alias you want:




\begin{pyin}
import sympy as a_poor_name_choice

a_poor_name_choice.cos(a_poor_name_choice.pi / 4)
\end{pyin}




\begin{equation*}
\begin{split}\displaystyle \frac{\sqrt{2}}{2}\end{split}
\end{equation*}

\begin{note}
\textbf{It is important} when aliasing to use accepted conventions for these
nicknames. For \texttt{sympy}, an accepted convention is indeed \texttt{import sympy as sym}.
\end{note}

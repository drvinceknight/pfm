\chapter{Matrices}
\label{chp:matrices}

Matrices form the building block of an area of mathematics referred to as Linear
Algebra. The dictionary definition of a matrix is:

\begin{quote}
``A group of numbers or other symbols arranged in a rectangle that can be used
together as a single unit to solve particular mathematical problems.''
\end{quote}


The particular mathematical problems referred to usually correspond to solving large
systems of linear equations. However they have become an area of interest in
their own right and manipulating matrices usually corresponds to:
\begin{itemize}
\item 

calculating the \index{determinant} of a matrix;

\item 

calculating the \index{inverse} of a matrix.

\end{itemize}


Here you will see how to instruct a computer to carry out these techniques.
In this chapter you will cover:

\begin{itemize}
\item 

Creating matrices.

\item 

Manipulating matrices.

\item 

Solving a system of linear equations using matrices.

\end{itemize}





\section{Tutorial}
\label{\detokenize{tools-for-mathematics/04-matrices/tutorial/main:tutorial}}\label{\detokenize{tools-for-mathematics/04-matrices/tutorial/main::doc}}

You will solve the following problem using a computer to assist with the technical aspects:

The matrix \(A\) is given by \(A=\begin{pmatrix}a & 1 & 1\\ 1 & a & 1\\ 1 & 1 & 2\end{pmatrix}\).
\begin{enumerate}

\item 

Find the \index{determinant} of \(A\)

\item 

Hence find the values of \(a\) for which \(A\) is singular.

\item 

For the following values of \(a\), when possible obtain \(A ^ {- 1}\) and confirm
the result by computing \(AA^{-1}\):
\begin{enumerate}

\item 

\(a = 0\);

\item 

\(a = 1\);

\item 

\(a = 2\);

\item 

\(a = 3\).

\end{enumerate}

\end{enumerate}



\texttt{sympy} is once again the library you will use for this.
You will start by defining the matrix \(A\):




\begin{pyin}
import sympy as sym

a = sym.symbol("a")
A = sym.matrix([[a, 1, 1], [1, a, 1], [1, 1, 2]])
\end{pyin}





You can now create a variable \texttt{\index{determinant}} and assign it the value of the
\index{determinant} of \(A\):




\begin{pyin}
determinant = A.det()
determinant
\end{pyin}




\begin{equation*}
\begin{split}\displaystyle 2 a^{2} - 2 a\end{split}
\end{equation*}




A matrix is singular if it has \index{determinant} 0. You can find the values of \(a\) for
which this occurs:




\begin{pyin}
sym.solveset(determinant, a)
\end{pyin}




\begin{equation*}
\begin{split}\displaystyle \left\{0, 1\right\}\end{split}
\end{equation*}

Thus, it is not possible to find the \index{inverse} of \(A\) for \(a\in\{0, 1\}\).
However for \(a = 2\):




\begin{pyin}
A.subs({a: 2})
\end{pyin}




\begin{equation*}
\begin{split}\displaystyle \left[\begin{matrix}2 & 1 & 1\\1 & 2 & 1\\1 & 1 & 2\end{matrix}\right]\end{split}
\end{equation*}






\begin{pyin}
A.subs({a: 2}).inv()
\end{pyin}




\begin{equation*}
\begin{split}\displaystyle \left[\begin{matrix}\frac{3}{4} & - \frac{1}{4} & - \frac{1}{4}\\- \frac{1}{4} & \frac{3}{4} & - \frac{1}{4}\\- \frac{1}{4} & - \frac{1}{4} & \frac{3}{4}\end{matrix}\right]\end{split}
\end{equation*}




To carry out matrix multiplication you use the \texttt{@} symbol:




\begin{pyin}
A.subs({a: 2}).inv() @ A.subs({a: 2})
\end{pyin}




\begin{equation*}
\begin{split}\displaystyle \left[\begin{matrix}1 & 0 & 0\\0 & 1 & 0\\0 & 0 & 1\end{matrix}\right]\end{split}
\end{equation*}




and for \(a = 3\):




\begin{pyin}
A.subs({a: 3}).inv()
\end{pyin}




\begin{equation*}
\begin{split}\displaystyle \left[\begin{matrix}\frac{5}{12} & - \frac{1}{12} & - \frac{1}{6}\\- \frac{1}{12} & \frac{5}{12} & - \frac{1}{6}\\- \frac{1}{6} & - \frac{1}{6} & \frac{2}{3}\end{matrix}\right]\end{split}
\end{equation*}






\begin{pyin}
A.subs({a: 3}).inv() @ A.subs({a: 3})
\end{pyin}




\begin{equation*}
\begin{split}\displaystyle \left[\begin{matrix}1 & 0 & 0\\0 & 1 & 0\\0 & 0 & 1\end{matrix}\right]\end{split}
\end{equation*}





\begin{note}
In this tutorial you have
\begin{itemize}
\item 

Created a matrix;

\item 

Calculated the \index{determinant} of the matrix;

\item 

Substituted values in the matrix;

\item 

Inverted the matrix.

\end{itemize}
\end{note}





\section{How to}
\label{\detokenize{tools-for-mathematics/04-matrices/how/main:how}}\label{\detokenize{tools-for-mathematics/04-matrices/how/main::doc}}

\subsection{Create a matrix}
\label{\detokenize{tools-for-mathematics/04-matrices/how/main:create-a-matrix}}

You create a matrix using the \texttt{sympy.Matrix} tool. Combine this with
nested square brackets \texttt{{[}{]}} so that every row is also inside square
brackets.


\begin{api}
sympy.Matrix([values])
\end{api}



For example, the following creates the matrix:
\begin{equation*}
\begin{split}
    B = \begin{pmatrix}
            1 & 2 & 3 & 4\\
            5 & 6 & 7 & 8\\
            9 & 10 & 11 & 12
        \end{pmatrix}
\end{split}
\end{equation*}



\begin{pyin}
import sympy as sym

B = sym.Matrix([[1, 2, 3, 4], [5, 6, 7, 8], [9, 10, 11, 12]])
B
\end{pyin}




\begin{equation*}
\begin{split}\displaystyle \left[\begin{matrix}1 & 2 & 3 & 4\\5 & 6 & 7 & 8\\9 & 10 & 11 & 12\end{matrix}\right]\end{split}
\end{equation*}

\begin{note}
It is possible to write the code in a more readable way as long as an incomplete
line ends with an open bracket:
\end{note}


\begin{pyin}
B = sym.Matrix(
    [
        [1, 2, 3, 4],
        [5, 6, 7, 8],
        [9, 10, 11, 12],
    ]
)
\end{pyin}


\subsection{Calculate the \index{determinant} of a matrix}
\label{\detokenize{tools-for-mathematics/04-matrices/how/main:calculate-the-determinant-of-a-matrix}}

To calculate the \index{determinant} of a matrix, use the \texttt{.det} tool.

\begin{api}
matrix = sympy.Matrix([values])
matrix.det()
\end{api}



For example, the \index{determinant} of the following matrix:
\begin{equation*}
\begin{split}
    \begin{pmatrix}
    1 & 5\\
    5 & 1
    \end{pmatrix}
\end{split}
\end{equation*}



\begin{pyin}
matrix = sym.Matrix([[1, 5], [5, 1]])
matrix.det()
\end{pyin}




\begin{equation*}
\begin{split}\displaystyle -24\end{split}
\end{equation*}




\subsection{Calculate the \index{inverse} of a matrix}
\label{\detokenize{tools-for-mathematics/04-matrices/how/main:calculate-the-inverse-of-a-matrix}}

To calculate the \index{inverse} of a matrix, use the \texttt{.inv} tool.


\begin{api}
matrix = sympy.Matrix([values])
matrix.inv()
\end{api}



For example to
calculate the \index{inverse} of:
\begin{equation*}
\begin{split}
    \begin{pmatrix}
        1 / 2 & 1\\
        5     & 0
    \end{pmatrix}
\end{split}
\end{equation*}



\begin{pyin}
matrix = sym.Matrix([[sym.S(1) / 2, 1], [5, 0]])
matrix.inv()
\end{pyin}




\begin{equation*}
\begin{split}\displaystyle \left[\begin{matrix}0 & \frac{1}{5}\\1 & - \frac{1}{10}\end{matrix}\right]\end{split}
\end{equation*}




\subsection{Multiply matrices by a scalar}
\label{\detokenize{tools-for-mathematics/04-matrices/how/main:multiply-matrices-by-a-scalar}}

To multiple a matrix by a scalar use the \texttt{*} operator. For example to
multiply the following matrix by \(6\):
\begin{equation*}
\begin{split}
    \begin{pmatrix}
        1 / 5 & 1\\
        1 & 1
    \end{pmatrix}
\end{split}
\end{equation*}



\begin{pyin}
matrix = sym.Matrix([[sym.S(1) / 5, 1], [1, 1]])
6 * matrix
\end{pyin}




\begin{equation*}
\begin{split}\displaystyle \left[\begin{matrix}\frac{6}{5} & 6\\6 & 6\end{matrix}\right]\end{split}
\end{equation*}




\subsection{Add matrices together}
\label{\detokenize{tools-for-mathematics/04-matrices/how/main:add-matrices-together}}

To add matrices use the \texttt{+} operator. For example to compute:
\begin{equation*}
\begin{split}
    \begin{pmatrix}
        1 / 5 & 1\\
        1 & 1
    \end{pmatrix} +
    \begin{pmatrix}
        4 / 5 & 0\\
        0 & 0
    \end{pmatrix}
\end{split}
\end{equation*}



\begin{pyin}
matrix = sym.Matrix([[sym.S(1) / 5, 1], [1, 1]])
other_matrix = sym.Matrix([[sym.S(4) / 5, 0], [0, 0]])
matrix + other_matrix
\end{pyin}




\begin{equation*}
\begin{split}\displaystyle \left[\begin{matrix}1 & 1\\1 & 1\end{matrix}\right]\end{split}
\end{equation*}




\subsection{Multiply matrices together}
\label{\detokenize{tools-for-mathematics/04-matrices/how/main:multiply-matrices-together}}

To multiply matrices together you use the \texttt{@} operator. For example to compute:
\begin{equation*}
\begin{split}
    \begin{pmatrix}
        1 / 5 & 1\\
        1 & 1
    \end{pmatrix}
    \begin{pmatrix}
        4 / 5 & 0\\
        0 & 0
    \end{pmatrix}
\end{split}
\end{equation*}



\begin{pyin}
matrix @ other_matrix
\end{pyin}




\begin{equation*}
\begin{split}\displaystyle \left[\begin{matrix}\frac{4}{25} & 0\\\frac{4}{5} & 0\end{matrix}\right]\end{split}
\end{equation*}




\subsection{How to create a vector}
\label{\detokenize{tools-for-mathematics/04-matrices/how/main:how-to-create-a-vector}}

A vector is essentially a matrix with a single row or column. For example to
create the vector:
\begin{equation*}
\begin{split}
    \begin{pmatrix}
    3 \\
    2 \\
    1
    \end{pmatrix}
\end{split}
\end{equation*}



\begin{pyin}
vector = sym.Matrix([[3], [2], [1]])
vector
\end{pyin}




\begin{equation*}
\begin{split}\displaystyle \left[\begin{matrix}3\\2\\1\end{matrix}\right]\end{split}
\end{equation*}




\subsection{How to solve a linear system}
\label{\detokenize{tools-for-mathematics/04-matrices/how/main:how-to-solve-a-linear-system}}

To solve a given linear system that can be represented in matrix form, create
the corresponding matrix and vector and the matrix \index{inverse}. For example to
solve the following equations:
\begin{equation*}
\begin{split}
    \begin{array}{l}
        x + 2y = 3\\
        3x + y + 2z = 4\\
        - y + z = 1\\
    \end{array}
\end{split}
\end{equation*}



\begin{pyin}
A = sym.Matrix([[1, 2, 0], [3, 1, 2], [0, -1, 1]])
b = sym.Matrix([[3], [4], [1]])
A.inv() @ b
\end{pyin}




\begin{equation*}
\begin{split}\displaystyle \left[\begin{matrix}- \frac{5}{3}\\\frac{7}{3}\\\frac{10}{3}\end{matrix}\right]\end{split}
\end{equation*}






\section{Exercises}
\label{\detokenize{tools-for-mathematics/04-matrices/exercises/main:exercises}}\label{\detokenize{tools-for-mathematics/04-matrices/exercises/main::doc}}

\begin{enumerate}

\item 

Obtain the \index{determinant} and the \index{inverse}s of the following matrices:
\begin{enumerate}

\item 

\(A = \begin{pmatrix} 1 / 5 & 1\\1 & 1\end{pmatrix}\)

\item 

\(B = \begin{pmatrix} 1 / 5 & 1 & 5\\3 & 1 & 6 \\ 1 & 2 & 1\end{pmatrix}\)

\item 

\(C = \begin{pmatrix} 1 / 5 & 5 & 5\\3 & 1 & 7 \\ a & b & c\end{pmatrix}\)

\end{enumerate}

\item 

Compute the following:
\begin{enumerate}

\item 

\(500\begin{pmatrix} 1 / 5 & 1\\1 & 1\end{pmatrix}\)

\item 

\(\pi \begin{pmatrix} 1 / \pi & 2\pi\\3/\pi & 1\end{pmatrix}\)

\item 

\(500\begin{pmatrix} 1 / 5 & 1\\1 & 1\end{pmatrix} + \pi \begin{pmatrix} 1 / \pi & 2\pi\\3/\pi & 1\end{pmatrix}\)

\item 

\(500\begin{pmatrix} 1 / 5 & 1\\1 & 1\end{pmatrix}\begin{pmatrix} 1 / \pi & 2\pi\\3/\pi & 1\end{pmatrix}\)

\end{enumerate}

\item 

The matrix \(A\) is given by \(A=\begin{pmatrix}a & 4 & 2\\ 1 & a & 0\\ 1 & 2 & 1\end{pmatrix}\).
\begin{enumerate}

\item 

Find the \index{determinant} of \(A\)

\item 

Hence find the values of \(a\) for which \(A\) is singular.

\item 

State, giving a brief reason in each case, whether the simultaneous equations
\begin{equation*}
\begin{split}
          \begin{array}{l}
              a x + 4y + 2z= 3a\\
              x + a y = 1\\
              x + 2y + z = 3\\
          \end{array}
      \end{split}
\end{equation*}

have any solutions when:
\begin{enumerate}

\item 

\(a = 3\);

\item 

\(a = 2\)

\end{enumerate}

\end{enumerate}

\item 

The matrix \(D\) is given by \(D = \begin{pmatrix} a & 2 & 0\\ 3 & 1 & 2\\ 0 & -1 & 1\end{pmatrix}\) where \(a\ne 2\).
\begin{enumerate}

\item 

Find \(D^{-1}\).

\item 

Hence or otherwise, solve the equations:

\end{enumerate}
\begin{equation*}
\begin{split}
   \begin{array}{l}
       a x + 2y = 3\\
       3x + y + 2z = 4\\
       - y + z = 1\\
   \end{array}
   \end{split}
\end{equation*}
\end{enumerate}


\section{Further information}
\label{\detokenize{tools-for-mathematics/04-matrices/why/main:further-information}}\label{\detokenize{tools-for-mathematics/04-matrices/why/main::doc}}

\subsection{Why does this book not discuss commenting of code?}
\label{\detokenize{tools-for-mathematics/04-matrices/why/main:why-are-we-not-commenting-our-code}}

In Python it is possible to write statements that are ignored using the \texttt{\#}
symbol. This creates something called a ``comment''. For example:




\begin{pyin}
import sympy as sym  # Importing the sympy library using an alias
\end{pyin}

Comments like these often do not add to the readability of the code. In fact
they can make the code less readable or at worse
confusing~\cite{martin2009clean}.

In this section of the book there is in fact no need for comments like this as
you are mainly using tools that are well documented. Furthermore when using
Jupyter notebooks you can add far more to the readability of the code by adding
prose alongside our code instead of using small brief inline comments.

This does not mean that readability of code is not important.


\begin{note}
Being able to read and understand written code is important.
\end{note}



In Chapter~\ref{chp:functions_and_data_structures}
you will start to write functions and emphasis will
be given there on readability and documenting (as opposed to commenting) the
code written. A specific discussion about using a tool called a
\textbf{docstring} as opposed to a comment will be covered.


In chapters~\ref{chp:modularisation} and~\ref{chp:testing} there is more
information on how to ensure code is readable and understandable.


\subsection{Why use \texttt{@} for matrix multiplication and not \texttt{*}?}
\label{\detokenize{tools-for-mathematics/04-matrices/why/main:why-do-we-use-for-matrix-multiplication-and-not}}

With \texttt{sympy} it is in fact possible to use the \texttt{*} operator for matrix
multiplication:

\begin{pyin}
import sympy as sym

matrix = sym.Matrix([[sym.S(1) / 5, 1], [1, 1]])
other_matrix = sym.Matrix([[sym.S(4) / 5, 0], [0, 0]])
matrix * other_matrix
\end{pyin}




\begin{equation*}
\begin{split}\displaystyle \left[\begin{matrix}\frac{4}{25} & 0\\\frac{4}{5} & 0\end{matrix}\right]\end{split}
\end{equation*}




However there are other libraries that can be used for linear algebra and in
those libraries the \texttt{*} does not do matrix multiplication, it does element wise
multiplication instead. So for clarity it is preferred to use \texttt{@} throughout.


\subsection{I have read that \texttt{numpy} is a library for linear algebra?}
\label{\detokenize{tools-for-mathematics/04-matrices/why/main:i-have-read-that-numpy-is-a-library-for-linear-algebra}}

\texttt{numpy} is one of the most popular and important libraries in the Python
ecosystem. It is in fact the best library to use when doing linear algebra as it
is computationally efficient, \textbf{however} it cannot handle symbolic variables
which is why you are seeing how to use \texttt{Sympy} here. An introduction to
\texttt{numpy} is covered in a chapter of the online version of the
book.

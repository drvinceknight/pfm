\chapter{Combinatorics}
\label{chp:combinatorics}

Combinatorics is the mathematical area interested in specific finite sets. In a lot
of instances this comes down to counting things and is often first encountered
by mathematicians through combinations and permutations.
Computers are useful when doing this as they can be used to generate the finite
sets considered. You can essentially count things ``by hand''' (using a computer)
to confirm theoretic results.


\begin{note}
In this chapter you will cover:
\begin{itemize}
\item 

Generating configurations of elements that correspond to permutations and/or
combinations.

\item 

Counting these configurations.

\item 

Directly computing \(^n C_i={n \choose i}\).

\item 

Directly computing \(^n P_i\).

\end{itemize}
\end{note}





\section{Tutorial}
\label{\detokenize{tools-for-mathematics/05-combinations-permutations/tutorial/main:tutorial}}\label{\detokenize{tools-for-mathematics/05-combinations-permutations/tutorial/main::doc}}

You will solve the following problem using a computer to illustrate how a
computer can be used to solve combinatorial problems:


The digits 1, 2, 3, 4 and 5 are arranged in random order, to form a five-digit number.
\begin{enumerate}

\item 

How many different five-digit numbers can be formed?

\item 

How many different five-digit numbers are:
\begin{enumerate}

\item 

Odd

\item 

Less than 23000

\end{enumerate}

\end{enumerate}



First you are going to get the 5 digits. Python has a tool for this called
\texttt{range} which directly gives the integers from a given bound to another:




\begin{pyin}
digits = range(1, 6)
digits
\end{pyin}





\begin{raw}
range(1, 6)
\end{raw}





At present that is only the instructions for generating the integers from 1 to 5
(the 6 is a strict upper bound). You can transform this to a tuple, using the
\texttt{tuple} tool:




\begin{pyin}
tuple(range(1, 6))
\end{pyin}





\begin{raw}
(1, 2, 3, 4, 5)
\end{raw}

The question is asking for all the permutations of size 5 of that set.
The main tool for this is a library specifically designed to iterate over
objects in different ways: \texttt{itertools}.

\begin{pyin}
import itertools

permutations = itertools.permutations(digits)
permutations
\end{pyin}





\begin{raw}
<itertools.permutations at 0x103a548b0>
\end{raw}

That is again only the set of instructions, to view the actual permutations you
will again transform this in to a tuple. You will overwrite the value
of \texttt{permutations} to not be the instructions but the actual tuple of all the
permutations:

\begin{pyin}
permutations = tuple(permutations)
permutations
\end{pyin}





\begin{raw}
((1, 2, 3, 4, 5),
 (1, 2, 3, 5, 4),
 (1, 2, 4, 3, 5),
 (1, 2, 4, 5, 3),
 ...
 (5, 4, 2, 3, 1),
 (5, 4, 3, 1, 2),
 (5, 4, 3, 2, 1))
\end{raw}






Now to answer the question you need to find out how many tuples are in that
tuple. You do this using the Python \texttt{len} tool which returns the length of
something:




\begin{pyin}
len(permutations)
\end{pyin}





\begin{raw}
120
\end{raw}





You can confirm this to be correct as you know that there are \(5!\) ways of
arranging those numbers. The \texttt{math} library has a \texttt{factorial} tool:




\begin{pyin}
import math

math.factorial(5)
\end{pyin}





\begin{raw}
120
\end{raw}





In order to find out how many 5 digit numbers are odd you are going to compute
the following sum:
\begin{equation*}
\begin{split}
    \sum_{\pi \in \Pi} \pi_5 \mod 2
\end{split}
\end{equation*}

Where \(\Pi\) is the set of permutations and \(\pi_5\) denotes the 5th (and last)
element of the permutation. So for example, if the first element of \(\Pi\) was
To do this you use the \texttt{sum} tool in python coupled with the expressions
\texttt{for}
and \texttt{in}. You also access the 5th element of a given \texttt{permutation}
using \texttt{[4]} (the
first element is indexed by 0, so the 5th is indexed by 4):




\begin{pyin}
sum(permutation[4] % 2 for permutation in permutations)
\end{pyin}





\begin{raw}
72
\end{raw}





You can again check this theoretically, there are three valid choices for the
last digit of a given tuple to be odd: \(1\), \(3\) and \(5\). For each of those,
there are then 4 choices for the remaining digits:




\begin{pyin}
math.factorial(4) * 3
\end{pyin}





\begin{raw}
72
\end{raw}

To compute the number of digits that are less than or equal to 23000 you compute a
similar sum except you use the \texttt{<=} operator and also convert the tuple of digits
to a number in base 10:
\begin{equation*}
\begin{split}
    (\pi_1, \pi_2, \pi_3, \pi_4, \pi_5) \to \pi_1 10 ^ 4 + \pi_2 10 ^ 3 + \pi_3 10 ^ 2 + \pi_4 10 + \pi_5
\end{split}
\end{equation*}

Thus you are going to compute the following sum:
\begin{equation*}
\begin{split}
    \sum_{\pi \in \Pi \text{ if }\pi_1 10 ^ 4 + \pi_2 10 ^ 3 + \pi_3 10 ^ 2 + \pi_4 10 + \pi_5 \leq 23000} 1
\end{split}
\end{equation*}



\begin{pyin}
sum(
    1
    for permutation in permutations
    if permutation[0] * 10 ** 4
    + permutation[1] * 10 ** 3
    + permutation[2] * 10 ** 2
    + permutation[3] * 10
    + permutation[4]
    <= 23000
)
\end{pyin}





\begin{raw}
30
\end{raw}





You can again confirm this theoretically, for a given tuple to be less than 23000
that is only possible if the first digit is 1 or 2:
\begin{itemize}
\item 

If it is 1 then any of the other \(4!\) permutations of the other digits is
valid;

\item 

If it is 2 then the second digit must be 1 and any of the other \(3!\)
permutations of the other digits is valid.

\end{itemize}

\begin{pyin}
(math.factorial(4) + math.factorial(3))
\end{pyin}

\begin{raw}
30
\end{raw}

\begin{note}
In this tutorial you have
\begin{itemize}
\item 

Created permutations of a given tuples;

\item 

Created permutations of a given tuples that obey a given condition;

\item 

Counted how many permutations exist;

\item 

Directly computed the expected number of permutations.

\end{itemize}
\end{note}





\section{How to}
\label{\detokenize{tools-for-mathematics/05-combinations-permutations/how/main:how}}\label{\detokenize{tools-for-mathematics/05-combinations-permutations/how/main::doc}}

\subsection{Create a tuple}
\label{sec:create_a_tuple}

To create a tuple which is an ordered collection of objects that cannot be
changed use the \texttt{()} brackets.


\begin{api}
collection = (value_1, value_2, value_3, ..., value_n)
\end{api}



For example:




\begin{pyin}
basket = ("Bread", "Biscuits", "Coffee")
basket
\end{pyin}





\begin{raw}
('Bread', 'Biscuits', 'Coffee')
\end{raw}

\subsection{How to access particular elements in a tuple}
\label{sec:how-to-access-particular-elements-in-a-tuple}

If you need to you can access elements of a collection using \texttt{[]} brackets. The
first element has index \texttt{0}:

\begin{api}
tuple[index]
\end{api}


For example:




\begin{pyin}
basket[1]
\end{pyin}





\begin{raw}
'Biscuits'
\end{raw}





\subsection{Create boolean variables}
\label{sec:create_boolean_variables}

A boolean variable has one of two values: \texttt{True} or \texttt{False}.


To create a boolean variable here are some of the things you can use:
\begin{itemize}
\item 

Equality: \texttt{value == other\_value}

\item 

Inequality \texttt{value != other\_value}

\item 

Strictly less than \texttt{value < other\_value}

\item 

Less than or equal\texttt{value <= other\_value}

\item 

Inclusion \texttt{value in iterable}

\end{itemize}


This a subset of the operators available.
For example:




\begin{pyin}
value = 5
other_value = 10

value == other_value
\end{pyin}





\begin{raw}
False
\end{raw}







\begin{pyin}
value != other_value
\end{pyin}





\begin{raw}
True
\end{raw}







\begin{pyin}
value <= other_value
\end{pyin}





\begin{raw}
True
\end{raw}







\begin{pyin}
value < value
\end{pyin}





\begin{raw}
False
\end{raw}







\begin{pyin}
value <= value
\end{pyin}





\begin{raw}
True
\end{raw}







\begin{pyin}
value in (1, 2, 4, 19)
\end{pyin}





\begin{raw}
False
\end{raw}





It is also possible to combine booleans to create new booleans:
\begin{itemize}
\item 

And: \texttt{first\_boolean and second\_boolean}

\item 

Or: \texttt{first\_boolean or second\_boolean}

\item 

No: \texttt{not boolean}

\end{itemize}




\begin{pyin}
True and True
\end{pyin}





\begin{raw}
True
\end{raw}







\begin{pyin}
False and True
\end{pyin}





\begin{raw}
False
\end{raw}







\begin{pyin}
True or False
\end{pyin}





\begin{raw}
True
\end{raw}







\begin{pyin}
False or False
\end{pyin}





\begin{raw}
False
\end{raw}







\begin{pyin}
not True
\end{pyin}





\begin{raw}
False
\end{raw}







\begin{pyin}
not False
\end{pyin}





\begin{raw}
True
\end{raw}





\subsection{Create an iterable with a given number of items}
\label{\detokenize{tools-for-mathematics/05-combinations-permutations/how/main:creating-an-iterable-with-a-given-number-of-items}}\label{\detokenize{tools-for-mathematics/05-combinations-permutations/how/main:id4}}

The \texttt{range} tool gives a given number of integers.


\begin{api}
range(number_of_integers)
\end{api}



For example:




\begin{pyin}
tuple(range(10))
\end{pyin}





\begin{raw}
(0, 1, 2, 3, 4, 5, 6, 7, 8, 9)
\end{raw}






\texttt{range(N)} gives the integers from 0 until \(N - 1\) (inclusive).



It is also possible to pass two values as inputs so that you have a different lower bound:




\begin{api}
tuple(range(4, 10))
\end{api}





\begin{raw}
(4, 5, 6, 7, 8, 9)
\end{raw}





It is also possible to pass a third value as a step size:




\begin{pyin}
tuple(range(4, 10, 3))
\end{pyin}





\begin{raw}
(4, 7)
\end{raw}





\subsection{Create permutations of a given set of elements}
\label{\detokenize{tools-for-mathematics/05-combinations-permutations/how/main:creating-permutations-of-a-given-set-of-elements}}

The python \texttt{itertools} library has a \texttt{permutations} tool that will generate all
permutations of a given set.


\begin{api}
itertools.permutations(iterable)
\end{api}





\begin{pyin}
import itertools

basket = ("Bread", "Biscuits", "Coffee")
tuple(itertools.permutations(basket))
\end{pyin}





\begin{raw}
(('Bread', 'Biscuits', 'Coffee'),
 ('Bread', 'Coffee', 'Biscuits'),
 ('Biscuits', 'Bread', 'Coffee'),
 ('Biscuits', 'Coffee', 'Bread'),
 ('Coffee', 'Bread', 'Biscuits'),
 ('Coffee', 'Biscuits', 'Bread'))
\end{raw}





It is possible to limit the size to only be permutations of size \texttt{r}:




\begin{pyin}
tuple(itertools.permutations(basket, r=2))
\end{pyin}





\begin{raw}
(('Bread', 'Biscuits'),
 ('Bread', 'Coffee'),
 ('Biscuits', 'Bread'),
 ('Biscuits', 'Coffee'),
 ('Coffee', 'Bread'),
 ('Coffee', 'Biscuits'))
\end{raw}





\subsection{Create combinations of a given set of elements}
\label{\detokenize{tools-for-mathematics/05-combinations-permutations/how/main:creating-combinations-of-a-given-set-of-elements}}

The python \texttt{itertools} library has a \texttt{combinations} tool that
will generate all combinations of size \texttt{r} of a given set:


\begin{api}
itertools.combinations(iterable, r)
\end{api}



For example:




\begin{pyin}
basket = ("Bread", "Biscuits", "Coffee")
tuple(itertools.combinations(basket, r=2))
\end{pyin}





\begin{raw}
(('Bread', 'Biscuits'), ('Bread', 'Coffee'), ('Biscuits', 'Coffee'))
\end{raw}





A combination does not care about order so by default the combinations generated
are sorted.


\subsection{Summing items in a iterable}

You can compute the sum of items in an iterable using the \texttt{sum} tool:




\begin{pyin}
sum((1, 2, 3))
\end{pyin}





\begin{raw}
6
\end{raw}





You can also directly use the \texttt{sum} without specifically creating the
iterable. This
corresponds to the following mathematical notation:
\begin{equation*}
\begin{split}
    \sum_{s\in S}f(s)
\end{split}
\end{equation*}

and is done using the following:

\begin{pyin}
sum(n ** 2 for n in range(11))
\end{pyin}


Here is an example of calculating the following sum:
\begin{equation*}
\begin{split}
    \sum_{n=0}^{10} n ^ 2
\end{split}
\end{equation*}



\begin{pyin}
sum(n ** 2 for n in range(11))
\end{pyin}





\begin{raw}
385
\end{raw}

You can compute conditional sums by only summing over elements that
meet a given condition using the following:

\begin{api}
sum(f(object) for object in old_list if condition)
\end{api}


Here is an example of calculating the following sum:
\begin{equation*}
\begin{split}
    \sum_{\begin{array}{c}n=0\\\text{ if }n\text{ odd}\end{array}}^{10} n ^ 2
\end{split}
\end{equation*}



\begin{pyin}
sum(n ** 2 for n in range(11) if n % 2 == 1)
\end{pyin}





\begin{raw}
165
\end{raw}





\subsection{Directly compute \(n!\)}
\label{\detokenize{tools-for-mathematics/05-combinations-permutations/how/main:directly-computing-n}}

The \texttt{math} library has a \texttt{factorial} tool.


\begin{api}
math.factorial(N)
\end{api}





\begin{pyin}
import math

math.factorial(5)
\end{pyin}





\begin{raw}
120
\end{raw}





\subsection{Directly compute \({n \choose i}\)}

The \texttt{scipy.special} library has a \texttt{comb} tool.


\begin{api}
scipy.special.comb(n, i)
\end{api}



For example:




\begin{pyin}
import scipy.special

scipy.special.comb(3, 2)
\end{pyin}





\begin{raw}
3.0
\end{raw}





\subsection{Directly compute \(^n P_i\)}
\label{\detokenize{tools-for-mathematics/05-combinations-permutations/how/main:directly-computing-n-p-i}}

The \texttt{scipy.special} library has a \texttt{perm} tool.


\begin{api}
scipy.special.perm(n, i)
\end{api}

For example:

\begin{pyin}
scipy.special.perm(3, 2)
\end{pyin}





\begin{raw}
6.0
\end{raw}







\section{Exercises}
\begin{enumerate}

\item 

Obtain the following tuples using the \texttt{range} command:
\begin{enumerate}

\item 

\((0, 1, 2, 3, 4, 5)\)

\item 

\((2, 3, 4, 5)\)

\item 

\((2, 4, 6, 8)\)

\item 

\(-1, 2, 5, 8\)

\end{enumerate}

\item 

By \textbf{both} generating and directly computing obtain the \textbf{number of} the following:
\begin{enumerate}

\item 

All permutations of \((0, 1, 2, 3, 4, 5)\).

\item 

All permutations of \(("A", "B", "C")\).

\item 

Permutations of size 3 of \((0, 1, 2, 3, 4, 5)\).

\item 

Permutations of size 2 of \((0, 1, 2, 3, 4, 5, 6)\).

\item 

Combinations of size 3 of \((0, 1, 2, 3, 4, 5)\).

\item 

Combinations of size 2 of \((0, 1, 2, 3, 4, 5)\).

\item 

Combinations of size 5 of \((0, 1, 2, 3, 4, 5)\).

\end{enumerate}

\item 

A class consists of 3 students from Ashville and 4 from Bewton. A committee of 5 students is chosen at random the class.
\begin{enumerate}

\item 

Find the number of committees that include 2 students from Ashville and 3 from Bewton are chosen.

\item 

In fact 2 students, from Ashville and 3 from Bewton are chosen. In order to watch a video, all 5 committee members sit in a row. In how many different orders can they sit if no two students from Bewton sit next to each other.

\end{enumerate}

\item 

Three letters are selected at random from the 8 letters of the word \texttt{COMPUTER}, without regard to order.
\begin{enumerate}

\item 

Find the number of possible selections of 3 letters.

\item 

Find the number of selections of 3 letters with the letter \texttt{P}.

\item 

Find the number of selections of 3 letters where the 3 letters form the word \texttt{TOP}.

\end{enumerate}

\end{enumerate}

\section{Further information}
\label{\detokenize{tools-for-mathematics/05-combinations-permutations/why/main:further-information}}\label{\detokenize{tools-for-mathematics/05-combinations-permutations/why/main::doc}}

\subsection{Are there other ways to access elements in tuples?}
\label{\detokenize{tools-for-mathematics/05-combinations-permutations/why/main:are-there-other-ways-to-access-elements-in-tuples}}

You have seen in this chapter how to access a single element in a tuple. There
are various ways of indexing tuples:
\begin{enumerate}

\item 

Indexing (seen in Section~\ref{sec:how-to-access-particular-elements-in-a-tuple}).

\item 

Negative indexing (see Section~\ref{sec:access_an_element_of_an_iterable_using_negative_indexing})

\item 

Slicing (see Section~\ref{sec:slice_an_iterable})

\end{enumerate}


\subsection{Why does \texttt{range}, \texttt{itertools.permutations} and \texttt{itertools.combinations} not directly give the elements?}

When you run either of the three \texttt{range}, \texttt{itertools.permutations} or
\texttt{itertools.combinations} tools this is an example of creating a \textbf{generator}.
This allows the creation of the instructions to build something without building
it.


In practice this means that you can create large sets without needing to generate
them until required.


\subsection{How does the summation notation \protect\(\sum\protect\) correspond to the code?}
\label{\detokenize{tools-for-mathematics/05-combinations-permutations/why/main:how-does-the-summation-notation-sum-correspond-to-the-code}}

The \texttt{sum} command corresponds to the mathematical
\(\sum\) notation. Here are a few examples showing the \texttt{sum} command,
the \(\sum\) notation but also the prose describing:

\begin{itemize}
\item 
\textbf{The sum of the square of the integers from 1 to 100 (inclusive)}:
        \[\sum_{i=1}^{100}i ^2\]

        Given by:

\begin{pyin}
sum(i ** 2 for i in range(1, 101))
\end{pyin}

\begin{raw}
338350
\end{raw}

\item
\textbf{The sum of the square of the integers from 1 to 100 (inclusive) if they
are prime}:
\[\sum_{\begin{array}{c}i=1\\\text{if }i\text{ is prime}\end{array}}^{100}i ^2\]
    Given by:
\begin{pyin}
sum(i ** 2 for i in range(1, 101) if sym.isprime(i))
\end{pyin}

\begin{raw}
65796
\end{raw}

\item 
\textbf{The sum of the square of the elements in the collection $S$ if they are
prime}:
\[\sum_{\begin{array}{c}i\in{S}\\\text{if }i\text{ is prime}\end{array}}i ^2\]
Given by:
\begin{pyin}
S = (1, 3, 9, 12, 21, 5, 2, 2)
sum(i ** 2 for i in S if sym.isprime(i))
\end{pyin}

\begin{raw}
42
\end{raw}
\end{itemize}

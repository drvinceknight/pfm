\chapter{Sequences}
\label{chp:sequences}

The formal definition of sequences is a collection of ordered objects with
potential repetitions.
The study of these sequences leads to many interesting results. Here you will
concentrate on using recursive definitions to generate the values in a sequence.

\begin{note}
In this chapter you will cover:
\begin{itemize}
\item 

Using \index{recursion}.

\end{itemize}
\end{note}




\section{Tutorial}
\label{\detokenize{tools-for-mathematics/07-sequences/tutorial/main:tutorial}}\label{\detokenize{tools-for-mathematics/07-sequences/tutorial/main::doc}}

You will solve the following problem using a computer using a programming
technique called \textbf{recursion}.


A sequence \(a_1, a_2, a_3, …\) is defined by:
\begin{equation*}
\begin{split}
    \left\{
    \begin{array}{l}
        a_1 = k,\\
        a_{n + 1} = 2a_n – 7, n \geq 1,
    \end{array}
    \right.
\end{split}
\end{equation*}

where \(k\) is a constant.
\begin{enumerate}

\item 

Write down an expression for \(a_2\) in terms of \(k\).

\item 

Show that \(a_3 = 4k -21\)

\item 

Given that \(\sum_{r=1}^4 a_r = 43\) find the value of \(k\).

\end{enumerate}



You will use Python to define a function that reproduces the mathematical
definition of \(a_k\):




\begin{pyin}
def generate_a(k_value, n):
    """
    Uses recursion to return a_n for a given value of k:

    a_1 = k
    a_n = 2a_{n-1} - 7
    """
    if n == 1:
        return k_value
    return 2 * generate_a(k_value, n - 1) - 7
\end{pyin}


\begin{note}
This is similar to the mathematical definition: the Python definition of
the function refers to itself.
\end{note}



You can use this to compute \(a_3\) for \(k=4\):




\begin{pyin}
generate_a(k_value=4, n=3)
\end{pyin}





\begin{raw}
-5
\end{raw}





You can use this to compute \(a_5\) for \(k=1\):




\begin{pyin}
generate_a(k_value=1, n=5)
\end{pyin}





\begin{raw}
-89
\end{raw}

Finally it is also possible to pass a symbolic value to \texttt{k\_value}. This allows
you to answer the first question:

\begin{pyin}
import sympy as sym

k = sym.Symbol("k")
generate_a(k_value=k, n=2)
\end{pyin}




\begin{equation*}
\begin{split}\displaystyle 2 k - 7\end{split}
\end{equation*}




Likewise for \(a_3\):




\begin{pyin}
generate_a(k_value=k, n=3)
\end{pyin}




\begin{equation*}
\begin{split}\displaystyle 4 k - 21\end{split}
\end{equation*}




For the last question start by computing the sum:
\begin{equation*}
\begin{split}
    \sum_{r=1}^4 a_r
\end{split}
\end{equation*}



\begin{pyin}
sum_of_first_four_terms = sum(generate_a(k_value=k, n=r) for r in range(1, 5))
sum_of_first_four_terms
\end{pyin}




\begin{equation*}
\begin{split}\displaystyle 15 k - 77\end{split}
\end{equation*}




This allows you to create the given equation and solve it:




\begin{pyin}
equation = sym.Eq(sum_of_first_four_terms, 43)
sym.solveset(equation, k)
\end{pyin}




\begin{equation*}
\begin{split}\displaystyle \left\{8\right\}\end{split}
\end{equation*}





\begin{note}
In this tutorial you have
\begin{itemize}
\item 

Defined a function using recursion.

\item 

Called this function using both numeric and symbolic values.

\end{itemize}
\end{note}





\section{How to}
\label{\detokenize{tools-for-mathematics/07-sequences/how/main:how}}\label{\detokenize{tools-for-mathematics/07-sequences/how/main::doc}}

\subsection{Define a function using recursion}
\label{\detokenize{tools-for-mathematics/07-sequences/how/main:define-a-function-using-recursion}}

It is possible to define a recursive expression using a recursive function in
Python. This requires two things:
\begin{itemize}
\item 

A recursive rule: something that relates the current term to another one;

\item 

A base case: a particular term that does not need the recursive rule to be
calculated.

\end{itemize}


Consider the following mathematical expression:
\begin{equation*}
\begin{split}
    \left\{
    \begin{array}{l}
        a_1 = 1,\\
        a_n = 2a_{n - 1}, n > 1,
    \end{array}
    \right.
\end{split}
\end{equation*}\begin{itemize}
\item 

The recursive rule is: \(a_n = 2a_{n - 1}\);

\item 

The base case is: \(a_1 = 1\).

\end{itemize}


In Python this can be written as:




\begin{pyin}
def generate_sequence(n):
    """
    Generate the sequence defined by:

    a_1 = 1
    a_n = 2 a_{n - 1}

    This is done using recursion.
    """
    if n == 1:
        return 1
    return 2 * generate_sequence(n - 1)
\end{pyin}





Here you can get the first 7 terms:




\begin{pyin}
values_of_sequence = [generate_sequence(n) for n in range(1, 8)]
values_of_sequence
\end{pyin}





\begin{raw}
[1, 2, 4, 8, 16, 32, 64]
\end{raw}







\section{Exercises}
\label{\detokenize{tools-for-mathematics/07-sequences/exercises/main:exercises}}\label{\detokenize{tools-for-mathematics/07-sequences/exercises/main::doc}}

\begin{enumerate}

\item 

Using recursion, obtain the first 10 terms of the following sequences:
\begin{enumerate}

\item 

\(\left\{\begin{array}{l}a_1 = 1,\\a_n = 3a_{n - 1}, n > 1\end{array}\right.\)

\item 

\(\left\{\begin{array}{l}b_1 = 3,\\b_n = 6b_{n - 1}, n > 1\end{array}\right.\)

\item 

\(\left\{\begin{array}{l}c_1 = 3,\\c_n = 6c_{n - 1} + 3, n > 1\end{array}\right.\)

\item 

\(\left\{\begin{array}{l}d_0 = 3,\\d_n = \sqrt{d_{n - 1}} + 3, n > 0\end{array}\right.\)

\end{enumerate}

\item 

Using recursion, obtain the first 5 terms of the \index{Fibonacci} sequence:
\begin{equation*}
\begin{split}
   \left\{
       \begin{array}{l}
           a_0 = 0,\\
           a_1 = 1,\\
           a_n = a_{n - 1} + a_{n - 2}, n \geq 2
       \end{array}
   \right.
   \end{split}
\end{equation*}
\item 

A 40 year building programme for new houses began in Oldtown in the year 1951 (Year 1) and finished in 1990 (Year 40).


The number of houses built each year form an arithmetic sequence with first term \(a\) and common difference \(d\).


Given that 2400 new houses were built in 1960 and 600 new houses were built in 1990, find:
\begin{enumerate}

\item 

The value of \(d\).

\item 

The value of \(a\).

\item 

The total number of houses built in Oldtown over 40 years.

\end{enumerate}

\item 

A sequence is given by:
\begin{equation*}
\begin{split}
       \left\{\begin{array}{l}
       x_1 = 1\\
       x_{n + 1}= x_n(p + x_n), n > 1
       \end{array}\right.
   \end{split}
\end{equation*}

for \(p\ne0\).
\begin{enumerate}

\item 

Find \(x_2\) in terms of \(p\).

\item 

Show that \(x_3=1+3p+2p^2\).

\item 

Given that \(x_3=1\), find the value of \(p\)

\end{enumerate}

\end{enumerate}

\section{Further information}
\label{\detokenize{tools-for-mathematics/07-sequences/why/main:further-information}}\label{\detokenize{tools-for-mathematics/07-sequences/why/main::doc}}

\subsection{What are the differences between recursion and iteration?}
\label{\detokenize{tools-for-mathematics/07-sequences/why/main:what-are-the-differences-between-recursion-and-iteration}}

When giving instructions to a computer it is possible to use recursion to
directly implement a common mathematical definition. For example consider the
following sequence:
\begin{equation*}
\begin{split}
    \left\{\begin{array}{l}
    a_1 = 1\\
    a_{n + 1}= 3a_n, n > 1
    \end{array}\right.
\end{split}
\end{equation*}

You can define this in Python as follows:

\begin{pyin}
def generate_sequence(n):
    """
    Generate the sequence defined by:

    a_1 = 1
    a_n = 3 a_{n - 1}

    This is done using recursion.
    """
    if n == 1:
        return 1
    return 3 * generate_sequence(n - 1)
\end{pyin}





The first 6 terms:




\begin{pyin}
[generate_sequence(n) for n in range(1, 7)]
\end{pyin}





\begin{raw}
[1, 3, 9, 27, 81, 243]
\end{raw}





In this case this corresponds to powers of \(3\), and indeed you can
prove that: \(a_n = 3 ^ {n - 1}\). The proof is not given here but one
approach to doing it would be to use induction which is closely related
to recursive functions.


You can write a different python function that uses this formula. This is called
\textbf{iteration}:




\begin{pyin}
def calculate_sequence(n):
    """
    Calculate the nth term of the sequence defined by:

    a_1 = 1
    a_n = 3 a_{n - 1}

    This is done using iteration using the direct formula:

    a_n = 3 ^ n
    """
    return 3 ** (n - 1)
\end{pyin}







\begin{pyin}
[calculate_sequence(n) for n in range(1, 7)]
\end{pyin}





\begin{raw}
[1, 3, 9, 27, 81, 243]
\end{raw}





You can in fact use a Jupyter command to
time the run time of a command. It is clear that recursion is slower.

\begin{pyin}
%timeit [generate_sequence(n) for n in range(1, 25)]
\end{pyin}





\begin{raw}
19.2 µs ± 246 ns per loop (mean ± std. dev. of 7 runs, 100,000 loops each)
\end{raw}











\begin{pyin}
%timeit [calculate_sequence(n) for n in range(1, 25)]
\end{pyin}





\begin{raw}
5.63 µs ± 44.7 ns per loop (mean ± std. dev. of 7 runs, 100,000 loops each)
\end{raw}







In practice:
\begin{itemize}
\item 

Using recursion is powerful as it can be used to directly implement recursive
definitions.

\item 

Using iteration is more computationally efficient but it is not always
straightforward to obtain an iterative formula.

\end{itemize}


\subsection{What is caching?}
\label{sec:what_is_caching}

One of the reasons that recursion is computationally inefficient is that it
always has to recalculate previously calculated values.


For example:
\begin{equation*}
\begin{split}
    \left\{\begin{array}{l}
    a_1 = 1\\
    a_{n + 1}= 3a_n, n > 1
    \end{array}\right.
\end{split}
\end{equation*}

One way to overcome this is to use caching which means that when a function is
called for a value it has already computed it remembers the value.
Python has a caching tool available in the \texttt{functools} library:




\begin{pyin}
import functools


def generate_sequence(n):
    """
    Generate the sequence defined by:

    a_1 = 1
    a_n = 3 a_{n - 1}

    This is done using recursion.
    """
    if n == 1:
        return 1
    return 3 * generate_sequence(n - 1)


@functools.lru_cache()
def cached_generate_sequence(n):
    """
    Generate the sequence defined by:

    a_1 = 1
    a_n = 3 a_{n - 1}

    This is done using recursion but also includes a cache.
    """
    if n == 1:
        return 1
    return 3 * cached_generate_sequence(n - 1)
\end{pyin}


Timing both these approaches confirms a substantial increase in time for the
cached version.

\begin{pyin}
%timeit [generate_sequence(n) for n in range(1, 25)]
\end{pyin}





\begin{raw}
20.5 µs ± 381 ns per loop (mean ± std. dev. of 7 runs, 100,000 loops each)
\end{raw}











\begin{pyin}
%timeit [cached_generate_sequence(n) for n in range(1, 25)]
\end{pyin}





\begin{raw}
934 ns ± 38.1 ns per loop (mean ± std. dev. of 7 runs, 1,000,000 loops each)
\end{raw}

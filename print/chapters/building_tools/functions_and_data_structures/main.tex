\chapter{Functions and data structures}
\label{chp:functions_and_data_structures}

In the previous chapters you have explored a number of tools that allow us to use
our mathematical knowledge more efficiently. In this chapter you 
continue to gain the knowledge necessary to build these tools covering the
following topics:



\begin{note}
In this chapter you will cover:
\begin{itemize}
\item 

Defining and using functions.

\item 

Defining and using various data structures.

\end{itemize}
\end{note}





\section{Tutorial}

Similarly to the Chapter~\ref{chp:variables_conditionals_and_loops}, you will use a computer to gain numerical
evidence for a problem.

The \index{Fibonacci} numbers are defined by the following sequence:
\begin{equation*}
\begin{split}
\left\{\begin{array}{l}
    a_0 = 0,\\
    a_1 = 1,\\
    a_n = a_{n - 1} + a_{n - 2}, n \geq 2\end{array}\right.
\end{split}
\end{equation*}

Verify that the following identity holds for \(n\leq 500\):
\begin{equation*}
\begin{split}
    \sum_{i=0}^n a_i = a_{n + 2} - 1
\end{split}
\end{equation*}


You will start by defining a function for \(a(n)\):




\begin{pyin}
import functools


@functools.lru_cache()
def get_fibonacci(n):
    """
    A function to give the nth Fibonacci number using the recursive
    definition.

    Note that this also uses a cache.

    Parameters
    ----------
    n: int
        The index of the Fibonacci number

    Returns
    -------
    int
        The nth Fibonacci number
    """
    if n == 0:
        return 0
    if n == 1:
        return 1
    return get_fibonacci(n - 1) + get_fibonacci(n - 2)
\end{pyin}

\begin{note}
This uses caching in the function definition with \texttt{lru\_cache}. This is not
necessary but makes the code more efficient. Caching is covered in
Section~\ref{sec:what_is_caching}.
\end{note}


You will print the first 10 numbers to ensure everything is working correctly:


\begin{pyin}
for n in range(10):
    print(get_fibonacci(n))
\end{pyin}





\begin{raw}
0
1
1
2
3
5
8
13
21
34
\end{raw}


Now write a function that returns a boolean: \texttt{True} if the equation
holds for a given value of \(n\), \texttt{False} otherwise.

\begin{pyin}
def check_theorem(n):
    """
    A function that generate the lhs and rhs of the
    following relationship:

    \sum_{i=0}^n a_i = a_{n + 2} - 1

    Where `a_i` is the i-th Fibonacci number.

    It checks if the relationship holds.

    Parameters
    ----------
    n: int
        The index n for which the theorem is to be verified.

    Returns
    -------
    bool
        Whether or not the theorem holds for a given n.
    """
    sum_of_fibonacci = sum(get_fibonacci(i) for i in range(n + 1))
    return sum_of_fibonacci == get_fibonacci(n + 2) - 1
\end{pyin}





Generate checks for \(n\leq 500\):





\begin{pyin}
checks = [check_theorem(n) for n in range(501)]
checks
\end{pyin}





\begin{raw}
[True,
 True,
 True,
 ...
 True,
 True,
 True]
\end{raw}






Confirm that all the booleans in \texttt{checks} are \texttt{True}:




\begin{pyin}
all(checks)
\end{pyin}





\begin{raw}
True
\end{raw}

\section{How to}
\label{\detokenize{building-tools/02-functions-and-data-structures/how/main:how}}\label{\detokenize{building-tools/02-functions-and-data-structures/how/main::doc}}

Two important data structures have already been seen in previous chapters:
\begin{itemize}
\item 

Tuples: Section~\ref{sec:create_a_tuple}.

\item 

Lists: Section~\ref{sec:create_a_list}.

\end{itemize}


\subsection{Define a function}

See: Section~\ref{sec:define_a_function}.


\subsection{Write a docstring}
\label{sec:write_a_docstring}

A docstring is an attribute of a function that describes what it is. This can
describe what it does, how it does it and/or why it does it.
Here is how to write a docstring for a function that takes variables and returns
a value.


\begin{api}
def name(parameter1, parameter2, ...):
    """
    <A description of what the function is.>

    Parameters
    ----------
    parameter1 : <type of parameter1>
        <description of parameter1>
    parameter2 : <type of parameter2>
        <description of parameter2>
    ...

    Returns
    -------
    <type of what the function returns>
        <description of what the function returns>

    """
    INDENTED BLOCK OF CODE
    return output
\end{api}



For example, here is how to write a function that returns \(x ^ 3\) for a given
\(x\):




\begin{pyin}
def x_cubed(x):
    """
    Calculates and returns the cube of x. Does this by using Python
    exponentiation.

    Parameters
    ----------
    x : float
        The value of x to be raised to the power 3

    Returns
    -------
    float
        The cube.
    """
    return x ** 3
\end{pyin}





\subsection{Create a tuple}

See: Section~\ref{sec:create_a_tuple}.



\subsection{Create a list}

See: Section~\ref{sec:create_a_list}.


\subsection{Create a list using a list comprehension}

See: Section~\ref{sec:create_a_list_using_a_list_comprehension}.


\subsection{Combine lists}

Given two lists it is possible to combine them to create a new list using the \texttt{+} operator:


\begin{api}
first_list + other_list
\end{api}



Here is an example creating a single list from two separate lists:




\begin{pyin}
first_list = [1, 2, 3]
other_list = [5, 6, 100]
combined_list = first_list + other_list
combined_list
\end{pyin}





\begin{raw}
[1, 2, 3, 5, 6, 100]
\end{raw}





\subsection{Append an element to a list}

Appending an element to a list is done using the \texttt{append} method.


\begin{api}
a_list.append(element)
\end{api}



Here is an example where you append a new string to a list of strings:




\begin{pyin}
names = ["Vince", "Zoe", "Julien", "Kaitlynn"]
names.append("Riggins")
names
\end{pyin}





\begin{raw}
['Vince', 'Zoe', 'Julien', 'Kaitlynn', 'Riggins']
\end{raw}

\begin{note}
It is not possible to do this with a \texttt{tuple} as a \texttt{tuple} is
\textbf{immutable}. See
Section~\ref{sec:what_is_the_difference_between_a_python_list_and_a_python_tuple}
for more information on the difference between a list and a tuple.
\end{note}



\subsection{Remove an element from a list}

To remove a given element from a list use the \texttt{remove} method.


\begin{api}
a_list.remove(element)
\end{api}



Here is an example removing a number from a list of numbers:




\begin{pyin}
numbers = [1, 94, 23, 202, 5]
numbers.remove(23)
numbers
\end{pyin}





\begin{raw}
[1, 94, 202, 5]
\end{raw}





\begin{note}
It is not possible to remove an element from a \texttt{tuple} as a
\texttt{tuple} is
immutable. See
Section~\ref{sec:what_is_the_difference_between_a_python_list_and_a_python_tuple}
for more information on the difference between a list and a tuple.

\end{note}



\subsection{Sort a list}
\label{\detokenize{building-tools/02-functions-and-data-structures/how/main:sort-a-list}}

To sort a list use the \texttt{sort} method.


\begin{api}
a_list.sort()
\end{api}



Here is an example:




\begin{raw}
names = ["Vince", "Zoe", "Kaitlynn", "Julien"]
names.sort()
names
\end{raw}





\begin{pyin}
names.sort(reverse=True)
names
\end{pyin}





To sort a list in reverse order use the \texttt{sort} method with the \texttt{reverse=True}
parameter.




\begin{pyin}
names.sort(reverse=True)
names
\end{pyin}





\begin{raw}
['Zoe', 'Vince', 'Kaitlynn', 'Julien']
\end{raw}






\begin{note}
It is not possible to sort a \texttt{tuple} as a \texttt{tuple} is immutable.
See
Section~\ref{sec:what_is_the_difference_between_a_python_list_and_a_python_tuple}
for more information on the difference between a list and a tuple.
\end{note}


\subsection{Create a sorted list from an iterable}
\label{\detokenize{building-tools/02-functions-and-data-structures/how/main:create-a-sorted-list-from-an-iterable}}

To create a sorted list from an iterable use the \texttt{sorted} function.


\begin{api}
sorted(iterable)
\end{api}



Here is an example:




\begin{pyin}
tuple_of_numbers = (20, 50, 10, 6, 1, 50, 105)
sorted(tuple_of_numbers)
\end{pyin}





\begin{raw}
[1, 6, 10, 20, 50, 50, 105]
\end{raw}





\subsection{Access an element of an iterable}
\label{\detokenize{building-tools/02-functions-and-data-structures/how/main:access-an-element-of-an-iterable}}

See: Section~\ref{sec:how-to-access-particular-elements-in-a-tuple}.


\subsection{Find the index of an element in an iterable}

To identify the position of an element in an iterable use the \texttt{index} method.


\begin{api}
iterable.index(element)
\end{api}



Here is an example:




\begin{pyin}
numbers = [1, 94, 23, 202, 5]
numbers.index(23)
\end{pyin}





\begin{raw}
2
\end{raw}


\begin{note}
Recall that python uses \texttt{0}-based indexing. The first element in an iterable has
index \texttt{0}.
\end{note}



\subsection{Access an element of an iterable using negative indexing}
\label{sec:access_an_element_of_an_iterable_using_negative_indexing}

It is possible to access an element of an iterable by counting from the end of
the iterable using negative indexing.


\begin{api}
iterable[-index_from_end]
\end{api}



Here is an example showing how to access the penultimate element in a tuple:




\begin{pyin}
basket = ("Carrots", "Potatoes", "Strawberries", "Juice", "Ice cream")
basket[-2]
\end{pyin}





\begin{raw}
'Juice'
\end{raw}





\subsection{Slice an iterable}
\label{sec:slice_an_iterable}

To create a new iterable from an iterable use \texttt{[]} and specify a start
(inclusive) and end (exclusive) pair of indices.


\begin{api}
iterable[include_start_index: exclusive_end_index]
\end{api}



For example:




\begin{pyin}
basket = ("Carrots", "Potatoes", "Strawberries", "Juice", "Ice cream")
basket[2: 5]
\end{pyin}





\begin{raw}
('Strawberries', 'Juice', 'Ice cream')
\end{raw}





\subsection{Find the number of elements in an iterable}
\label{\detokenize{building-tools/02-functions-and-data-structures/how/main:find-the-number-of-elements-in-an-iterable}}\label{\detokenize{building-tools/02-functions-and-data-structures/how/main:id3}}

To count the number of elements in an iterable use \texttt{len}:


\begin{api}
len(iterable)
\end{api}



For example:




\begin{pyin}
basket = ("Carrots", "Potatoes", "Strawberries", "Juice", "Ice cream")
len(basket)
\end{pyin}





\begin{raw}
5
\end{raw}





\subsection{Create a set}
\label{\detokenize{building-tools/02-functions-and-data-structures/how/main:create-a-set}}

A set is a collection of distinct objects. This can be created in Python using
the \texttt{set} command on any iterable. If there are non distinct objects in the
iterable then this is an efficient way to remove duplicates.


\begin{api}
set(iterable)
\end{api}



Here is an example of creating a set:




\begin{pyin}
iterable = (1, 1, 3, 4, 4, 3, 2, 1, 10)
unique_values = set(iterable)
unique_values
\end{pyin}





\begin{raw}
{1, 2, 3, 4, 10}
\end{raw}





\subsection{Do set operations}
\label{\detokenize{building-tools/02-functions-and-data-structures/how/main:do-set-operations}}

Set operations between two sets can be done using Python:
\begin{itemize}
\item 

\(S_1 \cup S_2\): \texttt{set\_1 | set\_2}

\item 

\(S_1 \cap S_2\): \texttt{set\_1 \& set\_2}

\item 

\(S_1 \setminus S_2\): \texttt{set\_1 - set\_2}

\item 

\(S_1 \subseteq S_2\) (checking if \(S_1\) is a subset of \(S_2\)): \texttt{set\_1 <= set\_2}

\end{itemize}


Here are some examples of carrying out the above:




\begin{pyin}
set_1 = set((1, 2, 3, 4, 5))
set_2 = set((4, 5, 6, 7, 8, 9))

set_1 | set_2
\end{pyin}





\begin{raw}
{1, 2, 3, 4, 5, 6, 7, 8, 9}
\end{raw}







\begin{pyin}
set_1 & set_2
\end{pyin}





\begin{raw}
{4, 5}
\end{raw}







\begin{pyin}
set_1 - set_2
\end{pyin}





\begin{raw}
{1, 2, 3}
\end{raw}







\begin{pyin}
set_1 <= set_2
\end{pyin}





\begin{raw}
False
\end{raw}





\subsection{Create hash tables}
\label{sec:create_hash_tables}

Lists and tuples allow us to immediately recover a value given its position.
Hash tables allow us to create arbitrary \texttt{key} \texttt{value} pairs so that given any
\texttt{key} you can immediately recover the value. This is called a dictionary in
Python and is created using \texttt{\{\}} which takes a collection of \texttt{key: value}
pairs.


\begin{api}
{key_1: value, key_2: value, ...}
\end{api}



For example the following dictionary maps pet names to their ages:




\begin{pyin}
ages = {"Riggins": 4, "Chick": 7, "Duck": 7}
ages
\end{pyin}





\begin{raw}
{'Riggins': 4, 'Chick': 7, 'Duck': 7}
\end{raw}





To recover a value pass the key to the dictionary using \texttt{{[}{]}}.


For example:




\begin{pyin}
ages["Riggins"]
\end{pyin}





\begin{raw}
4
\end{raw}





\begin{note}
If a key is used to recover the value with \texttt{{[}{]}} but the key is not in the dictionary then
an error will be raised.
\end{note}



\subsection{Access element in a hash table}
\label{\detokenize{building-tools/02-functions-and-data-structures/how/main:access-element-in-a-hash-table}}

As described in Section\ref{sec:create_hash_tables} to access the value of a \texttt{key} in a hash
table use \texttt{{[}{]}}.


\begin{api}
dictionary[key]
\end{api}



It is also possible to use the \texttt{get} method.
The \texttt{get} method can also be passed the value of a \texttt{default} variable to return when the
\texttt{key} is not in the hash table:


\begin{pyin}
dictionary.get(key, default)
\end{pyin}



For example:




\begin{pyin}
ages = {"Riggins": 4, "Chick": 7, "Duck": 7}
ages.get("Vince", -1)
\end{pyin}





\begin{raw}
-1
\end{raw}





\subsection{Iterate over keys in a hash table}
\label{\detokenize{building-tools/02-functions-and-data-structures/how/main:iterate-over-keys-in-a-hash-table}}

To iterate over the keys in a hash table use the \texttt{keys()} method:


\begin{api}
dictionary.keys()
\end{api}



For example:




\begin{pyin}
ages = {"Riggins": 4, "Chick": 7, "Duck": 7}
ages.keys()
\end{pyin}





\begin{raw}
dict_keys(['Riggins', 'Chick', 'Duck'])
\end{raw}





\subsection{Iterate over values in a hash table}
\label{\detokenize{building-tools/02-functions-and-data-structures/how/main:iterate-over-values-in-a-hash-table}}

To iterate over the values in a hash table use the \texttt{values()} method:


\begin{api}
dictionary.values()
\end{api}



For example:




\begin{pyin}
ages = {"Riggins": 4, "Chick": 7, "Duck": 7}
ages.values()
\end{pyin}





\begin{pyin}
dict_values([4, 7, 7])
\end{pyin}





\subsection{Iterate over pairs of keys and value in a hash table}
\label{\detokenize{building-tools/02-functions-and-data-structures/how/main:iterate-over-pairs-of-keys-and-value-in-a-hash-table}}

To iterate over pairs of keys and values in a hash table use the \texttt{items()} method:


\begin{api}
dictionary.items()
\end{api}



For example:




\begin{pyin}
ages = {"Riggins": 4, "Chick": 7, "Duck": 7}
ages.items()
\end{pyin}





\begin{raw}
dict_items([('Riggins', 4), ('Chick', 7), ('Duck', 7)])
\end{raw}







\section{Exercises}
\begin{enumerate}

\item 

Write a function that generates \(n!\).

\item 

Write a function that generates the \(n\)th triangular numbers defined by:
\begin{equation*}
\begin{split}
       T_n = \frac{n(n+1)}{2}
   \end{split}
\end{equation*}
\item 

Verify the following that the following identify holds for positive integer
values \(n\leq 500\):
\begin{equation*}
\begin{split}
       \sum_{i=0}^n T_i = \frac{n(n+1)(n+2)}{6}
   \end{split}
\end{equation*}
\item 

Consider the \textbf{Monty Hall problem} \cite{selvin1975monty}:

\begin{quote}
``Suppose you're on a game show, and you're given the choice of three doors:
Behind one door is a car; behind the others, goats. You pick a door, say No. 1,
and the host, who knows what's behind the doors, opens another door, say No. 3,
which has a goat. He then says to you, `Do you want to pick door No. 2?' Is it
to your advantage to switch your choice?''
\end{quote}

\begin{enumerate}

\item 

Write a function that simulates the play of the game when you ‘stick’ with
the initial choice. You might find \texttt{random.shuffle} and \texttt{pop}ing a list
helpful.

\item 

Write a function that simulates the play of the game when you ‘change’
your choice. You might find \texttt{remove}ing from a list helpful.

\item 

Repeat the play of the game using both those functions and compare the
probability of winning.

\end{enumerate}

\end{enumerate}




\section{Further information}
\label{\detokenize{building-tools/02-functions-and-data-structures/why/main:further-information}}\label{\detokenize{building-tools/02-functions-and-data-structures/why/main::doc}}

\subsection{What formats can be used to write a docstring?}
\label{\detokenize{building-tools/02-functions-and-data-structures/why/main:what-formats-can-be-used-to-write-a-docstring}}

The format used to write a docstring described in 
Section~\ref{sec:write_a_docstring}.
is the one specified
by the Numpy project. 


Amongst other things you can see how to specify further functionality:
\begin{itemize}
\item 

How to indicate if a parameter is optional.

\item 

How to specify what types of errors might be raised by a function.

\item 

How to specify when a function is a generator.

\end{itemize}


There are 2 other common specifications:
\begin{itemize}
\item 

Google’s Python Style Guide.

\item 

Sphinx Python Style Guide.

\end{itemize}


\subsection{Are there tools available to assist with writing docstrings?}
\label{\detokenize{building-tools/02-functions-and-data-structures/why/main:are-there-tools-available-to-assist-with-writing-docstrings}}

The \texttt{darglint} library can be used
to check if docstrings match a given format.


\subsection{Apart from removing duplicates and set operations what are they advantages to using \texttt{set}?}
\label{\detokenize{building-tools/02-functions-and-data-structures/why/main:a-part-from-removing-duplicates-and-set-operations-what-are-they-advantages-to-using-set}}

One valuable uses of \texttt{set} is to efficiently identify if an element
is in a given iterable or not:

\begin{pyin}
numbers = list(range(100000))
%timeit 100000 in numbers
\end{pyin}





\begin{raw}
474 µs ± 2.51 µs per loop (mean ± std. dev. of 7 runs, 1,000 loops each)
\end{raw}



\begin{pyin}
numbers = set(range(100000))
%timeit 100000 in numbers
\end{pyin}





\begin{raw}
15.2 ns ± 0.121 ns per loop (mean ± std. dev. of 7 runs, 100,000,000 loops each)
\end{raw}

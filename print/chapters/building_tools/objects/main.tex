\chapter{Object oriented programming}
\label{chp:objects}

In the first part of this book you covered a number of tools that allow you to
carry out mathematical techniques. One example of this is the \texttt{sympy.Symbol}
object that creates a symbolic variable that can be manipulated. In this chapter
you will see how to define similar mathematical objects.


\begin{note}
In this chapter you will cover:
\begin{itemize}
\item 

Creating objects.

\item 

Giving objects attributes.

\item 

Defining methods on objects.

\item 

Inheriting new objects from others.

\end{itemize}
\end{note}





\section{Tutorial}
\label{sec:objects_tutorial}

You will write some code to create and manipulate \index{quadratic} expressions.
With \texttt{sympy} this is not necessary as all functionality required is available
within \texttt{sympy} however this will be a good exercise in understanding how to
build such functionality.


Consider the following quadratics:
\begin{equation*}
\begin{split}
f(x) = 5 x ^ 2 + 2 x - 7\\
g(x) = - 4 x ^ 2 - 3 x + 12\\
h(x) = f(x) + g(x)
\end{split}
\end{equation*}

Without using \texttt{sympy}, obtain the roots for all the quadratics.

Start by defining an object to represent a \index{quadratic}. This is called a
class.


\begin{pyin}
import math


class QuadraticExpression:
    """A class for a quadratic expression"""

    def __init__(self, a, b, c):
        self.a = a
        self.b = b
        self.c = c
        self.discriminant = self.b ** 2 - 4 * self.a * self.c

    def get_roots(self):
        """
        Return the real valued roots of the quadratic expression

        Returns
        -------
        array
            The roots of the quadratic
        """
        if self.discriminant >= 0:
            x1 = -(self.b + math.sqrt(self.discriminant)) / (2 * self.a)
            x2 = -(self.b - math.sqrt(self.discriminant)) / (2 * self.a)
            return x1, x2
        return ()

    def __add__(self, other):
        """A magic method: let's us have addition between expressions"""
        return QuadraticExpression(self.a + other.a, self.b + other.b, self.c + other.c)

    def __repr__(self):
        """A magic method: changes the default way an instance is displayed"""
        return f"Quadratic expression: {self.a} x ^ 2 + {self.b} x + {self.c}"
\end{pyin}

\begin{note}
Four functions were created with this class:
\begin{itemize}
\item 

\texttt{\_\_init\_\_}: as this is surrounded by \texttt{\_\_} (two underscores) this is a magic
function that is run when we create an instance of our class.

\item 

\texttt{get\_roots}: this returns the two real valued roots if the
\index{discriminant} is
positive.

\item 

\texttt{\_\_add\_\_}: a magic function that is run when the \texttt{+} operator is used.

\item 

\texttt{\_\_repr\_\_}: a magic function that gives the string representation of the
instance.

\end{itemize}
\end{note}


Now use this class to solve the specified problem. First create
instances of the class that correspond to \(f\) and \(g\). This is using the \texttt{\_\_init\_\_}
function in the background.




\begin{pyin}
f = QuadraticExpression(a=5, b=2, c=-7)
g = QuadraticExpression(a=-4, b=-3, c=12)
\end{pyin}





You can now take a look at both of these instances. This is using the \texttt{\_\_repr\_\_}
function in the background:




\begin{pyin}
f
\end{pyin}





\begin{raw}
Quadratic expression: 5 x ^ 2 + 2 x + -7
\end{raw}







\begin{pyin}
g
\end{pyin}





\begin{raw}
Quadratic expression: -4 x ^ 2 + -3 x + 12
\end{raw}





Now you are going to create \(h(x) = f(x) + g(x)\). This is using the \texttt{\_\_add\_\_}
function in the background:




\begin{pyin}
h = f + g
h
\end{pyin}





\begin{raw}
Quadratic expression: 1 x ^ 2 + -1 x + 5
\end{raw}





You can now iterate over the quadratics and find the roots. This is using the
\texttt{get\_roots} function in the background:





\begin{pyin}
roots = [quadratic.get_roots() for quadratic in (f, g, h)]
roots
\end{pyin}





\begin{raw}
[(-1.4, 1.0), (1.3971808598447282, -2.1471808598447284), ()]
\end{raw}






Note that \(f\) and \(g\) have real valued roots but \(h\) does not. You can check
the value of the discriminant of \(h\):




\begin{pyin}
h.discriminant
\end{pyin}





\begin{raw}
-19
\end{raw}





You are going to create a new class from \texttt{QuadraticExpression} 
replacing the \texttt{get\_roots} function with a new one that can handle imaginary roots
and update the \texttt{\_\_add\_\_} function to return an instance of the new
class.




\begin{pyin}
class QuadraticExpressionWithAllRoots(QuadraticExpression):
    """
    A class for a quadratic expression that can return imaginary roots

    The `get_roots` function returns two tuples of the form (re, im) where re is
    the real part and im is the imaginary part.
    """

    def get_roots(self):
        """
        Return the real valued roots of the quadratic expression

        Returns
        -------
        array
            The roots of the quadratic
        """
        if self.discriminant >= 0:
            x1 = -(self.b + math.sqrt(self.discriminant)) / (2 * self.a)
            x2 = -(self.b - math.sqrt(self.discriminant)) / (2 * self.a)
            return (x1, 0), (x2, 0)

        real_part = self.b / (2 * self.a)
        im1 = math.sqrt(-self.discriminant) / (2 * self.a)
        im2 = -math.sqrt(-self.discriminant) / (2 * self.a)
        return ((real_part, im1), (real_part, im2))

    def __add__(self, other):
        """A special method: let's us have addition between expressions"""
        return QuadraticExpressionWithAllRoots(
            self.a + other.a, self.b + other.b, self.c + other.c
        )
\end{pyin}


Now define the quadratics once again but using this new class:




\begin{pyin}
f = QuadraticExpressionWithAllRoots(a=5, b=2, c=-7)
g = QuadraticExpressionWithAllRoots(a=-4, b=-3, c=12)
h = f + g
\end{pyin}







\begin{pyin}
f
\end{pyin}





\begin{raw}
Quadratic expression: 5 x ^ 2 + 2 x + -7
\end{raw}







\begin{pyin}
g
\end{pyin}





\begin{raw}
Quadratic expression: -4 x ^ 2 + -3 x + 12
\end{raw}







\begin{pyin}
h
\end{pyin}





\begin{pyin}
Quadratic expression: 1 x ^ 2 + -1 x + 5
\end{pyin}

\begin{note}
There is no need to redefine \texttt{\_\_init\_\_}, or \texttt{\_\_repr\_\_} as the new
class inherits these from \texttt{QuadraticExpression} due to this statement:
\end{note}

\begin{pyin}
class QuadraticExpressionWithAllRoots(QuadraticExpression):
\end{pyin}



You can now get all the roots for the quadratics:





\begin{pyin}
roots = [quadratic.get_roots() for quadratic in (f, g, h)]
roots
\end{pyin}





\begin{raw}
[((-1.4, 0), (1.0, 0)),
 ((1.3971808598447282, 0), (-2.1471808598447284, 0)),
 ((-0.5, 2.179449471770337), (-0.5, -2.179449471770337))]
\end{raw}








\section{How to}

\subsection{Define a class}

Define a class using the \texttt{class} keyword:


\begin{api}
class Name:
    """
    A docstring between triple quotation to describe what the class represents
    """
    INDENTED BLOCKS OF CODE
\end{api}



For example to create a class for a country:




\begin{pyin}
class Country:
    """
    A class to represent a country
    """
\end{pyin}





\subsection{Create an instance of the class}

Once a class is defined call it using the \texttt{()}:


\begin{api}
Name()
\end{api}



For example:





\begin{pyin}
first_country = Country()
first_country
\end{pyin}





\begin{raw}
<__main__.Country at 0x7f22a8f76e00>
\end{raw}









\begin{pyin}
second_country = Country()
second_country
\end{pyin}





\begin{raw}
<__main__.Country at 0x7f22a8f76e30>
\end{raw}


The \texttt{at ...>} is a pointer to the location of the instance in memory. If you re run
the code that location will change.


\subsection{Create an attribute}

Attributes are variables that belong to instances of classes. There can be
created and accessed using \texttt{.name\_of\_variable}.


For example, the following creates the attributes \texttt{name} and
\texttt{amount\_of\_magic}:




\begin{pyin}
first_country.name = "narnia"
first_country.amount_of_magic = 500
\end{pyin}





You can access them:




\begin{pyin}
first_country.name
\end{pyin}





\begin{raw}
'Narnia'
\end{raw}







\begin{pyin}
first_country.amount_of_magic
\end{pyin}





\begin{raw}
500
\end{raw}





You can manipulate them in place:




\begin{pyin}
first_country.amount_of_magic += 100
first_country.amount_of_magic
\end{pyin}





\begin{raw}
600
\end{raw}





\subsection{Create and call a method}
\label{\detokenize{building-tools/03-objects/how/main:how-to-create-and-call-a-method}}

Methods are functions that belong to classes. Define a function using the
\texttt{def} keyword (short for define). The first variable of a method is always the
specific instance that will call the method (it is passed implicitly).


\begin{api}
class Name:
    """
    A docstring between triple quotation to describe what the class represents
    """
    def name(self, parameter1, parameter2, ...):
        """
        <A description of what the method is.>

        Parameters
        ----------
        parameter1 : <type of parameter1>
            <description of parameter1>
        parameter2 : <type of parameter2>
            <description of parameter2>
        ...

        Returns
        -------
        <type of what the function returns>
            <description of what the function returns>

        """
        INDENTED BLOCK OF CODE
        return output
\end{api}



For example let us create a class for a country that has the ability to
``spend'''
magic:




\begin{pyin}
class Country:
    """
    A class to represent a country
    """

    def spend_magic(self, amount_spent):
        """
        Updates the magic attribute by subtracting amount_spent

        Parameters
        ----------
        amount_spent : float
            The amount of mana used.
        """
        self.amount_of_magic -= amount_spent
\end{pyin}

Now use it:

\begin{pyin}
first_country = Country()
first_country.name = "Narnia"
first_country.amount_of_magic = 500
first_country.spend_magic(amount_spent=100)
first_country.amount_of_magic
\end{pyin}





\begin{raw}
400
\end{raw}


\begin{note}
Even though the method is defined as taking two variables as inputs: \texttt{self} and
\texttt{amount\_spent} you only have to explicitly pass it \texttt{amount\_spent}. The first
variable in a method definition always corresponds to the instance on which the
method exists.
\end{note}



\subsection{How to create and call magic methods}
\label{sec:how_to_create_and_call_magic_methods}

Some methods can be called in certain situations:
\begin{itemize}
\item 

When creating an instance.

\item 

When wanting to display an instance.

\end{itemize}


These are referred to as dunder methods as they are all in between two
underscores: \texttt{\_\_}.


The method that is called when an instance is created is called \texttt{\_\_init\_\_} (for
initialised). For example:


\begin{api}
class Country:
    """
    A class to represent a country
    """

    def __init__(self, name, amount_of_magic):
        self.name = name
        self.amount_of_magic = amount_of_magic
\end{api}





Now instead of creating an instance and then creating the attributes you can do
those two things at the same time, by passing the variables to the class itself
(which in turn passes them to the \texttt{\_\_init\_\_} method):




\begin{pyin}
first_country = Country("Narnia", 500)
first_country.name
\end{pyin}





\begin{raw}
'Narnia'
\end{raw}


\begin{pyin}
first_country.amount_of_magic
\end{pyin}





\begin{raw}
500
\end{raw}





The method that returns a representation of an instance is \texttt{\_\_repr\_\_}. For
example:




\begin{pyin}
class Country:
    """
    A class to represent a country
    """

    def __init__(self, name, amount_of_magic):
        self.name = name
        self.amount_of_magic = amount_of_magic

    def __repr__(self):
        """Returns a string representation of the instance"""
        return f"{self.name} with {self.amount_of_magic} magic."
\end{pyin}







\begin{pyin}
first_country = Country("Narnia", 500)
first_country
\end{pyin}





\begin{raw}
Narnia with 500 magic.
\end{raw}

There are numerous other magic methods, such as the \texttt{\_\_add\_\_} one used in
the tutorial of this Chapter.

\subsection{Use inheritance}
\label{sec:how_to_use_inheritance}

Inheritance is a tool that allows you to create one class based on another. This
is done by passing the \texttt{Old} class to the \texttt{New} class.

\begin{api}
class New(Old)
\end{api}

For example let us create a class of \texttt{MuggleCountry} that overwrites the
\texttt{\_\_init\_\_} method but keeps the \texttt{\_\_rep\_\_} method of the
\texttt{Country} class written
in Section~\ref{sec:how_to_create_and_call_magic_methods}:

\begin{pyin}
class MuggleCountry(Country):
    """
    A class to represent a country with no magic. It only requires the name on
    initialisation.
    """

    def __init__(self, name):
        self.name = name
        self.amount_of_magic = 0
\end{pyin}

This has replaced the \texttt{\_\_init\_\_} method but the \texttt{\_\_repr\_\_} method is the same:

\begin{pyin}
other_country = MuggleCountry("Wales")
other_country
\end{pyin}

\begin{raw}
Wales with 0 magic.
\end{raw}

\section{Exercises}

\begin{enumerate}

\item 

Use the class created in Section~\ref{sec:objects_tutorial} to find the roots of the
following quadratics:
\begin{enumerate}

\item 

\(f(x) = -4x ^ 2 + x + 6\)

\item 

\(g(x) = 3x ^ 2 - 6\)

\item 

\(h(x) = f(x) + g(x)\)

\end{enumerate}

\item 

Write a class for a \index{linear expression} and use it to find the roots of the
following expressions:
\begin{enumerate}

\item 

\(f(x) = 2x + 6\)

\item 

\(g(x) = 3x - 6\)

\item 

\(h(x) = f(x) + g(x)\)

\end{enumerate}

\item 

If rain drops were to fall randomly on a square of side length \(2r\) the
probability of the drops landing in an inscribed circle of radius \(r\) would
be given by:
\begin{equation*}
\begin{split}
       P = \frac{\text{Area of circle}}{\text{Area of square}}=\frac{\pi r ^2}{4r^2}=\frac{\pi}{4}
   \end{split}
\end{equation*}

Thus, if you can approximate \(P\) then you can approximate \(\pi\) as \(4P\). In this
question you will write code to approximate \(P\) using the random library.


First create the following class:

\begin{pyin}
class Drop:
    """
    A class used to represent a random rain drop falling on a square of
    length r.
    """

    def __init__(self, r=1):
        self.x = (0.5 - random.random()) * 2 * r
        self.y = (0.5 - random.random()) * 2 * r
        self.in_circle = (self.y) ** 2 + (self.x) ** 2 <= r ** 2
\end{pyin}


Note that the above uses the following equation for a circle centred at
\((0,0)\) of radius \(r\):
\begin{equation*}
\begin{split}
       x^2+y^2\leq r^2
   \end{split}
\end{equation*}

To approximate \(P\) create \(N=1000\) instances of Drops and count the
number of those that are in the circle. Use this to approximate \(\pi\).

\item 

In a similar fashion to question 3, approximate the integral
\(\int_{0}^11-x^2\;dx\). Recall that the integral corresponds to the area
under a curve.

\end{enumerate}




\section{Further information}

\subsection{How to pronounce the double underscore?}

The double underscore used in magic methods like \texttt{\_\_init\_\_} or
\texttt{\_\_repr\_\_} is
pronounced ``dunder''.


\subsection{What is the \texttt{self} variable for?}

In methods the first variable is used to refer to the instance of a given class.
It is conventional to use \texttt{self}.


As an example consider this class:




\begin{pyin}
class PetDog:
    """
    A class for a Pet.

    Has two methods:
        - `bark` which returns "Woof" as a string.
        - `give_toy` which gives a toy to the dog in question. This updates the
          `toys` attribute.
    """

    def __init__(self):
        self.toys = []

    def bark(self):
        """
        Returns the string Woof.
        """
        return "Woof"

    def give_toy(self, toy):
        """
        Updates the instances toys list.
        """
        self.toys.append(toy)
\end{pyin}





If you now create two dogs:




\begin{pyin}
auraya = PetDog()
riggins = PetDog()
\end{pyin}





Both have no toys:




\begin{pyin}
auraya.toys
\end{pyin}





\begin{raw}
[]
\end{raw}







\begin{pyin}
riggins.toys
\end{pyin}





\begin{raw}
[]
\end{raw}





Now when you want to give \texttt{riggins} a toy you need to specify which of those two
empty lists to update:




\begin{pyin}
riggins.give_toy("ball")
riggins.toys
\end{pyin}





\begin{raw}
['ball']
\end{raw}

However \texttt{auraya} still has no toys:

\begin{pyin}
auraya.toys
\end{pyin}





\begin{raw}
[]
\end{raw}





When running \texttt{riggins.give\_toy("ball")}, internally the \texttt{give\_toy} method is
taking \texttt{self} to be \texttt{riggins} and so the
line \texttt{self.toys.append(toy)} in fact is running as \texttt{riggins.toys.append(toy)}.


The variable name \texttt{self} is a convention and not a functional requirement.
If you modify it
(for example by using inheritance as explained in
Section~\ref{sec:how_to_use_inheritance}.




\begin{pyin}
class OtherPetDog(PetDog):
    """
    A class for a Pet.

    Has two methods:
        - `bark` which returns "Woof" as a string.
        - `give_toy` which gives a toy to the dog in question. This updates the
          `toys` attribute.
    """

    def give_toy(the_dog_in_question, toy):
        """
        Updates the instances toys list.
        """
        the_dog_in_question.toys.append(toy)
\end{pyin}





Then you get the same outcome:




\begin{pyin}
riggins = OtherPetDog()
riggins.toys
\end{pyin}





\begin{raw}
[]
\end{raw}







\begin{pyin}
riggins.give_toy("ball")
riggins.toys
\end{pyin}





\begin{raw}
['ball']
\end{raw}





Indeed the line \texttt{the\_dog\_in\_question.toys.append(toy)} is run as
\texttt{riggins.toys.append(toy)}.


You should however use \texttt{self} as it is convention and helps with readability of
your code.


\subsection{Why use \texttt{CamelCase} for classes but \texttt{snake\_case} for functions?}

This is specified by the Python convention which is called PEP8~\cite{pep8}.


These conventions are important as it helps with readability of code.

\subsection{What is the difference between a method and a function?}

A method is a function defined on a class and always takes a first parameter
which is the specific instance from which the method is called.
